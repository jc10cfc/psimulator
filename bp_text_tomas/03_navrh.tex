\chapter{Analysa a návrh aplikace}


\section{Požadavky na aplikaci}

Nejprve shrnu všechny požadavky na mojí aplikaci.

\subsection{Funkční požadavky}
\begin{enumerate}
 \item Vytvoření počítačové sítě založené na počítačích OS Linux.
 \item Aplikace umožňuje konfiguraci rozhraní pomocí příkazů ifconfig a ip addr.
 \item Aplikace obsahuje funkční směrování a umožňuje jeho nastavování pomocí příkazů route a ip route.
 \item Aplikace implementuje překlad adres
 \item Aplikace podporuje ukládání a načítání do/ze souboru.
 \item Pro ověření správnosti jsou implementovány příkazy ping a traceroute.
 \item K jednotlivým počítačům aplikace je možné se připojit pomocí telnetu.
\end{enumerate}

\subsection{Nefunkční požadavky}
\begin{enumerate}
 \item Aplikace bude multiplatformní - alespoň pro operační systémy Windows a Linux
 \item Aplikace musí být spustitelná na běžném\footnote{Slovem \uv{běžné} se myslí v podstatě jakýkoliv počítač, na kterém je možné nainstalovat prostředí Javy - Java Runtime Environment} studentském počítači.
 \item Aplikace by měla být co nejvěrnější kopií reálného počítače s Linuxem.
\end{enumerate}


\section{Analysa požadavků}

\subsection{Připojení pomocí telnetu}

Jedním z funkčních požadavků mé aplikace je možnost připojit se k jednotlivým virtuálním počítačům pomocí protokolu telnet. Tento požadavek vypadá jednoduše, pokud pod pojmem Telnet chápeme jednoduchý protokol na přenos textových dat. Takový protokol ovšem neumožňuje doplňování příkazů a jejich historii, což je pro práci s počítačem, byť virtuálním, obrovské omezení. Oproti tomu, implementovat telnet protokol, jako NVT\footnote{NVT – Network Virtual Terminal, česky: Síťový virtuální terminál; poskytuje standardní rozhraní příkazové řádky}, kde se posílá a potvrzuje každý napsaný znak, by překračovalo rozsah této bakalářské práce. Proto jsem se rozhodl implementovat jen první možnost a na straně klienta řešit doplňování příkazů a jejich historii pomocí programu rlwrap, který funguje pod linuxem nebo přes cygwin i pod windows. Spouštění programu přes cygwin ve windows je sice velkou nevýhodou, ale neměl jsem jinou možnost. I tak ovšem základní požadavek, že s aplikací bude možno komunikovat pomocí telnetu, zůstává zachován, uživatel ovšem přijde o komfort, který mu nabízí možnost doplňování, editace a historie příkazů. 

\subsection{Podobnost simulátoru se skutečným linuxem}

Aby byl simulátor využitelný pro výukové účely, musí být dostatečně podobbný skutečnému linuxu, aby uživatel mohl věřit, že to, co funguje v simulátoru, bude fungovat i na skutečném linuxu a naopak. K tomu je ale potřeba implementovat jen ty příkazy, kterými se nastavují síťové parametry, a jen v takovém rozsahu, jaký je pro tyto výukové účely potřeba. Nezabývám se tedy kompletním příkazem ifconfig nebo ip, ale jen tou jejich částí, kterou se nastavují parametry rozhraní, jako IP, maska a další. O statních parametrech pak vetšinou simulátor vypíše, že ve skutečnosti sice existují, ale simulátorem zatím nejsou podporované.


\section{Programovací jazyk a uživatelské rozhraní}

\subsection{Programovací jazyk}
Aplikaci jsem se rozhodl programovat v programovacím jazyku Java z několika důvodů. Java je programovací jazyk, který nabízí velký programátorský komfort, stabilitu a zároveň možnost vytvořené aplikace používat pod různými operačními systémy, což je další z nefunkčních požadavků. Tento jazyk navíc disponuje hotovými knihovnami pro práci se sítí v balíčku java.net. Dalším důvodem je také to, že s programováním aplikací v Javě mám zatím asi největší zkušenosti.

\subsection{Uživatelské rozhraní}

Jak plyne ze zadání, uživatel se bude k jednotlivým virtuálním počítačům přihlašovat pomocí programu telnet, nemusím tedy vytvářet žádného speciálního klienta. S aplikací samotnou nebude uživatel nijak pracovat, jenom ji spustí se správným konfiguračním souborem a případně číslem výchozího portu, dále již bude nastavovat pouze jednotlivé virtuální počítače pomocí telnetu. Pro takovou aplikaci je nejlepším uživatelským rozhraním příkazová řádka, vytváření grafického uživatelského rozhraní by nemělo smysl.


\section{Průběh implementace}

Na aplikaci musím spolupracovat se svým kolegou Stanislavem Řehákem, který implementuje její druhou část - simulátor Cisca. Spolupráce však přesahuje jen tuto oblast a zasahuje také do společného jádra aplikace.



%% History:
% Pavel Tvrdik (26.12.2004)
%  + initial version for PhD Report
%
% Daniel Sykora (27.01.2005)
%
% Michal Valenta (3.12.2008)
% rada zmen ve formatovani (diky M. Duškovi, J. Holubovi a J. Žďárkovi)
% sjednoceni zdrojoveho kodu pro anglickou, ceskou, bakalarskou a diplomovou praci

% One-page layout: (proof-)reading on display
%%%% \documentclass[11pt,oneside,a4paper]{book}
% Two-page layout: final printing
\documentclass[11pt,twoside,a4paper]{book}   
%=-=-=-=-=-=-=-=-=-=-=-=--=%
% The user of this template may find useful to have an alternative to these 
% officially suggested packages:
\usepackage[czech, english]{babel}
\usepackage[T1]{fontenc} % pouzije EC fonty 
% pripadne pisete-li cesky, pak lze zkusit take:
% \usepackage[OT1]{fontenc} 
\usepackage[utf8]{inputenc}
%=-=-=-=-=-=-=-=-=-=-=-=--=%
% In case of problems with PDF fonts, one may try to uncomment this line:
%\usepackage{lmodern}
%=-=-=-=-=-=-=-=-=-=-=-=--=%
%=-=-=-=-=-=-=-=-=-=-=-=--=%
% Depending on your particular TeX distribution and version of conversion tools 
% (dvips/dvipdf/ps2pdf), some (advanced | desperate) users may prefer to use 
% different settings.
% Please uncomment the following style and use your CSLaTeX (cslatex/pdfcslatex) 
% to process your work. Note however, this file is in UTF-8 and a conversion to 
% your native encoding may be required. Some settings below depend on babel 
% macros and should also be modified. See \selectlanguage \iflanguage.
%\usepackage{czech}  %%%%%\usepackage[T1]{czech} %%%%[IL2] [T1] [OT1]
%=-=-=-=-=-=-=-=-=-=-=-=--=%

%%%%%%%%%%%%%%%%%%%%%%%%%%%%%%%%%%%%%%%
% Styles required in your work follow %
%%%%%%%%%%%%%%%%%%%%%%%%%%%%%%%%%%%%%%%
\usepackage{graphicx}
%\usepackage{indentfirst} %1. odstavec jako v cestine.

\usepackage{k336_thesis_macros} % specialni makra pro formatovani DP a BP
 % muzete si vytvorit i sva vlastni v souboru k336_thesis_macros.sty
 % najdete  radu jednoduchych definic, ktere zde ani nejsou pouzity
 % napriklad: 
 % \newcommand{\bfig}{\begin{figure}\begin{center}}
 % \newcommand{\efig}{\end{center}\end{figure}}
 % umoznuje pouzit prikaz \bfig namisto \begin{figure}\begin{center} atd.


%-----------------------------------------------------------------------------------------------------------------------------
% Tady jsou moje vlastní úpravy: 

% 2 verse správnýho nadpisu paragraphu, vůbec nevím, jak to funguje
% stazeno z: http://www.latex-community.org/forum/viewtopic.php?f=5&t=1383&start=0&st=0&sk=t&sd=a
% \makeatletter
% \renewcommand\paragraph{%
%    \@startsection{paragraph}{5}{0mm} % díky pětce se to nečísluje, ty milimetry jsou odsazaní do strany
%       {-\baselineskip}%
%       {.1\baselineskip} %tohle značí nadpis na samostatný řádce
%       {\normalfont\normalsize\bfseries}}
% \makeatother
\makeatletter
\renewcommand\paragraph{\@startsection{paragraph}{4}{\z@}%
  {-1ex\@plus -1ex \@minus -.1ex}%
  {1ex \@plus .1ex}%
  {\normalfont\normalsize\bfseries}}
\makeatother
%-----------------------------------------------------------------------------------------------------------------------------

%%%%%%%%%%%%%%%%%%%%%%%%%%%%%%%%%%%%%
% Zvolte jednu z moznosti 
% Choose one of the following options
%%%%%%%%%%%%%%%%%%%%%%%%%%%%%%%%%%%%%
% \newcommand\TypeOfWork{Diplomová práce} \typeout{Diplomova prace}
% \newcommand\TypeOfWork{Master's Thesis}   \typeout{Master's Thesis} 
\newcommand\TypeOfWork{Bakalářská práce}  \typeout{Bakalarska prace}
% \newcommand\TypeOfWork{Bachelor's Project}  \typeout{Bachelor's Project}


%%%%%%%%%%%%%%%%%%%%%%%%%%%%%%%%%%%%%
% Zvolte jednu z moznosti 
% Choose one of the following options
%%%%%%%%%%%%%%%%%%%%%%%%%%%%%%%%%%%%%
% nabidky jsou z: http://www.fel.cvut.cz/cz/education/bk/prehled.html

%\newcommand\StudProgram{Elektrotechnika a informatika, dobíhající, Bakalářský}
%\newcommand\StudProgram{Elektrotechnika a informatika, dobíhající, Magisterský}
% \newcommand\StudProgram{Elektrotechnika a informatika, strukturovaný, Bakalářský}
%  \newcommand\StudProgram{Elektrotechnika a informatika, strukturovaný, Navazující magisterský}
\newcommand\StudProgram{Softwarové technologie a management, Bakalářský}
% English study:
% \newcommand\StudProgram{Electrical Engineering and Information Technology}  % bachelor programe
% \newcommand\StudProgram{Electrical Engineering and Information Technology}  %master program


%%%%%%%%%%%%%%%%%%%%%%%%%%%%%%%%%%%%%
% Zvolte jednu z moznosti 
% Choose one of the following options
%%%%%%%%%%%%%%%%%%%%%%%%%%%%%%%%%%%%%
% nabidky jsou z: http://www.fel.cvut.cz/cz/education/bk/prehled.html

%\newcommand\StudBranch{Výpočetní technika}   % pro program EaI bak. (dobihajici i strukt.)
% \newcommand\StudBranch{Výpočetní technika}   % pro prgoram EaI mag. (dobihajici i strukt.)
\newcommand\StudBranch{Softwarové inženýrství}            %pro STM
%\newcommand\StudBranch{Web a multimedia}                  % pro STM
%\newcommand\StudBranch{Computer Engineering}              % bachelor programe
%\newcommand\StudBranch{Computer Science and Engineering}  % master programe


%%%%%%%%%%%%%%%%%%%%%%%%%%%%%%%%%%%%%%%%%%%%
% Vyplnte nazev prace, autora a vedouciho
% Set up Work Title, Author and Supervisor
%%%%%%%%%%%%%%%%%%%%%%%%%%%%%%%%%%%%%%%%%%%%

\newcommand\WorkTitle{Simulátor virtuální počítačové sítě Linux}
\newcommand\FirstandFamilyName{Tomáš Pitřinec}
\newcommand\Supervisor{Ing. Pavel Kubalík, Ph.D.}


% Pouzijete-li pdflatex, tak je prijemne, kdyz bude mit vase prace
% funkcni odkazy i v pdf formatu
\usepackage[
pdftitle={\WorkTitle},
pdfauthor={\FirstandFamilyName},
bookmarks=true,
colorlinks=true,
breaklinks=true,
urlcolor=red,
citecolor=blue,
linkcolor=blue,
unicode=true,
]
{hyperref}



% Extension posted by Petr Dlouhy in order for better sources reference (\cite{} command) especially in Czech.
% April 2010
% See comment over \thebibliography command for details.

\usepackage[square, numbers]{natbib}             % sazba pouzite literatury
%\usepackage{url}
%\DeclareUrlCommand\url{\def\UrlLeft{<}\def\UrlRight{>}\urlstyle{tt}}  %rm/sf/tt
%\renewcommand{\emph}[1]{\textsl{#1}}    % melo by byt kurziva nebo sklonene,
\let\oldUrl\url
\renewcommand\url[1]{<\texttt{\oldUrl{#1}}>}




\begin{document}

%%%%%%%%%%%%%%%%%%%%%%%%%%%%%%%%%%%%%
% Zvolte jednu z moznosti 
%%%%%%%%%%%%%%%%%%%%%%%%%%%%%%%%%%%%%
\selectlanguage{czech}
%\selectlanguage{english} 


% prikaz \typeout vypise vyse uvedena nastaveni v prikazovem okne
% pro pohodlne ladeni prace
\iflanguage{czech}{
	 \typeout{************************************************}
	 \typeout{Zvoleny jazyk: cestina}
	 \typeout{Typ prace: \TypeOfWork}
	 \typeout{Studijni program: \StudProgram}
	 \typeout{Obor: \StudBranch}
	 \typeout{Jmeno: \FirstandFamilyName}
	 \typeout{Nazev prace: \WorkTitle}
	 \typeout{Vedouci prace: \Supervisor}
	 \typeout{***************************************************}
	 \newcommand\Department{Katedra počítačů}
	 \newcommand\Faculty{Fakulta elektrotechnická}
	 \newcommand\University{České vysoké učení technické v Praze}
	 \newcommand\labelSupervisor{Vedoucí práce}
	 \newcommand\labelStudProgram{Studijní program}
	 \newcommand\labelStudBranch{Obor}
}{
	 \typeout{************************************************}
	 \typeout{Language: english}
	 \typeout{Type of Work: \TypeOfWork}
	 \typeout{Study Program: \StudProgram}
	 \typeout{Study Branch: \StudBranch}
	 \typeout{Author: \FirstandFamilyName}
	 \typeout{Title: \WorkTitle}
	 \typeout{Supervisor: \Supervisor}
	 \typeout{***************************************************}
	 \newcommand\Department{Department of Computer Science and Engineering}
	 \newcommand\Faculty{Faculty of Electrical Engineering}
	 \newcommand\University{Czech Technical University in Prague}
	 \newcommand\labelSupervisor{Supervisor}
	 \newcommand\labelStudProgram{Study Programme} 
	 \newcommand\labelStudBranch{Field of Study}
}




%%%%%%%%%%%%%%%%%%%%%%%%%%    Poznamky ke kompletaci prace
% Nasledujici pasaz uzavrenou v {} ve sve praci samozrejme 
% zakomentujte nebo odstrante. 
% Ve vysledne svazane praci bude nahrazena skutecnym 
% oficialnim zadanim vasi prace.
{
\pagenumbering{roman} \cleardoublepage \thispagestyle{empty}
\chapter*{Na tomto místě bude oficiální zadání vaší práce}
\begin{itemize}
\item Toto zadání je podepsané děkanem a vedoucím katedry,
\item musíte si ho vyzvednout na studiijním oddělení Katedry počítačů na Karlově náměstí,
\item v jedné odevzdané práci bude originál tohoto zadání (originál zůstává po obhajobě na katedře),
\item ve druhé bude na stejném místě neověřená kopie tohoto dokumentu (tato se vám vrátí po obhajobě).
\end{itemize}
\newpage
}

%%%%%%%%%%%%%%%%%%%%%%%%%%    Titulni stranka / Title page 

\coverpagestarts

%%%%%%%%%%%%%%%%%%%%%%%%%%%    Podekovani / Acknowledgements 

\acknowledgements
\noindent
Chtěl bych poděkovat panu Ing. Pavlu Kubalíkovi Ph.D. za dobrý námět k bakalářké práci a za pomoc při její realizaci. Dále bych chtěl poděkovat své rodině a své slečně za podporu a zázemí při psaní této práce i během celého studia.


%%%%%%%%%%%%%%%%%%%%%%%%%%%   Prohlaseni / Declaration 
% text toho prohlášení se odněkud includuje

\declaration{V~Červeném Kostelci dne 25.\,5.\,2010}

%%%%%%%%%%%%%%%%%%%%%%%%%%%%    Abstract 
 
\abstractpage

% Standuv abstrakt:
% This bachelor thesis inquires into desing and implementation of virtual network simulator composed of cisco-based routers for subject Y36PSI Počítačové sítě. Simulator allows configuration of routers via Cisco IOS command line. Simulator implements commands for configuring interfaces, static routing, dynamic and static network address translation. This system also offers verification of current configuration via ping and traceroute commands. Entire network settings are loaded from a configuration file and there is also possibility to save current configuration to a given file. User testing is integral part of a thesis.

% muj abstrakt moji angličtinou:
% This bachelor thesis inquires into design, implementation and testing of simulator of computer network based on computers with Linux OS. Simulator is made for use in subject Y36PSI, Computer networks, on FEE CTU in Prague. The application allows to build virtual computer network consist of computers with OS Linux, to configure the network with common commands used in linux command line, to save configurated network to a given file and also load the configuration from this file.

% muj abstrakt Standovou angličtinou:
This bachelor thesis inquires into design, implementation and testing of computer network simulator based on computers with
Linux OS. Simulator is intended to be used in subject Y36PSI, Computer networks, on FEE CTU in Prague. The application allows
building virtual computer network consisted of computers with OS Linux, configuring the network with common linux commands, saving
current configuration to a given file and loading back.

% Prace v cestine musi krome abstraktu v anglictine obsahovat i
% abstrakt v cestine.
\vglue60mm

\noindent{\Huge \textbf{Abstrakt}}
\vskip 2.75\baselineskip

\noindent
Tato bakalářská práce se zabývá analýzou, implementací a testováním jednoduchého simulátoru počítačové sítě založené po počítačích s OS Linux. Simulátor má sloužit především pro výukové účely předmětu Y36PSI, Počítačové sítě, na Fakultě elektrotechnické ČVUT v Praze. Aplikace umožňuje postavit virtuální počítačovou síť z počítačů s OS Linux, konfigurovat tuto síť běžnými příkazy používanými na linuxu v příkazové řádce, ukládat nakonfigurovanou síť do souboru a znovu ji z něho načítat.


%%%%%%%%%%%%%%%%%%%%%%%%%%%%%%%%  Obsah / Table of Contents 

\tableofcontents


%%%%%%%%%%%%%%%%%%%%%%%%%%%%%%%  Seznam obrazku / List of Figures 

\listoffigures

%%%%%%%%%%%%%%%%%%%%%%%%%%%%%%%  Seznam tabulek / List of Tables

% \listoftables žádnou tabulku nemám, tak jsem to zrušil

%**************************************************************

\mainbodystarts
% horizontalní mezera mezi dvema odstavci
%\parskip=5pt
%11.12.2008 parskip + tolerance
\normalfont
\parskip=0.2\baselineskip plus 0.2\baselineskip minus 0.1\baselineskip

% Odsazeni prvniho radku odstavce resi class book (neaplikuje se na prvni 
% odstavce kapitol, sekci, podsekci atd.) Viz usepackage{indentfirst}.
% Chcete-li selektivne zamezit odsazeni 1. radku nektereho odstavce,
% pouzijte prikaz \noindent.

%**************************************************************

% vkládání pomoci prikazu \include{jmeno_souboru.tex} nebo \include{jmeno_souboru}.
% napr: \chapter{Úvod}

% Úvod charakterizující kontext zadání, případně motivace.
% ----------
% Navrhněte a~implementujte aplikaci, která umožní vytvoření virtuální počítačové sítě, pro potřeby předmětu Y36PSI. Na
% systém se bude možno připojit s~pomocí telnetu. Z~pohledu uživatele se bude systém tvářit jako reálná síť. Zaměřte se
% především na konfiguraci systému Cisco. Systém bude podporovat příkazy potřebné ke konfiguraci síťových rozhraní,
% směrování a~překladu adres. Pro ověření správnosti konfigurace implementujte příkaz ping a~traceroute.
% ----------

Úkolem této práce je navrhnout a implementovat aplikaci, která umožní vytvoření virtuální počítačové sítě pro předmět Y36PSI\footnote{Počítačové sítě}. Z pohledu uživatele se systém musí tvářit jako reálná síť. Tento úkol byl rozdělen na dvě části: Cisco a Linux. Můj úkol je právě emulace Cisco IOS\footnote{Internetwork Operating System je operační systém používaných na směrovačích a přepínačích firmy Cisco Systems}. Na dnešním virtuálním trhu existuje celá řada programů pro virtualizaci sítě. Většina z nich je však špatně dostupných (zejména kvůli licenci) nebo se nehodí pro potřeby předmětu Počítačové sítě. 

Vize je taková, že student si v teple domova spustí tuto aplikaci a \uv{pohraje} si s virtuálním ciscem, ke kterému běžný smrtelník nemá přístup. Zjistí, jak to funguje a pak už jen přijde na cvičení předmětu a vše nakonfiguruje tak, jak to má být. 

Jelikož tento projekt přesahuje rozsah jedné bakalářské práce, tak byly vymezeny hranice, aby se tento úkol mohl rozdělit na dvě části. Nakonec celá aplikace byla rozdělena na části tři. První je část společná, kde je implementováno jádro server - klient. Druhá část je Cisco IOS, tu jsem dostal na starost já\footnote{Oba jsme chtěli programovat linuxovou část, protože s OS Linux máme oba zkušenosti. Po zralém losování \uv{Černý Petr = Cisco} padlo na mne.}. A třetí část je platforma Linux, kterou zpracoval Tomáš Pitřinec.



% Aplikace musí být jednoduše spustitelná pod běžným studentským počítačem.

\chapter{Úvod}

% pokyny k mojí práci:
% Navrhněte a implementujte aplikaci, která umožní vytvoření virtuální počítačové sítě, pro potřeby předmětu Y36PSI. Na systém se bude možno připojit s pomocí telnetu. Z pohledu uživatele se bude systém tvářit jako reálná síť. Zaměřte se především na konfiguraci systému Linux. Systém bude podporovat příkazy potřebné ke konfiguraci síťových rozhraní, směrování a překladu adres. Pro ověření správnosti konfigurace implementujte příkaz ping a traceroute.


% původní dlouhej úvod:
% Jednou z laboratorních úloh předmětu Počítačové sítě (Y36PSI) na FELu je postavení sítě mezi několika linuxovými a ciscovými počítači. Studenti, kteří tento úkol vykonávají, nemají často s takovou činností žádnou osobní zkušenost. Z přednášek většinou vědí, jak by měli síť očíslovat, avšak jen několik málo uživatelů linuxu umí nastavit adresy a brány na počítači s linuxem a jen nepatrný počet studentů nastavoval síťové adresy na skutečném Cisco routeru. Studenti pak při laboratořích řeší různé banální problémy, kterým by se mohli vyhnout, kdyby měli možnost zkusit si nastavit podobou síť již před laboratorní úlohou. Kvůli kapacitě laboratoře však není možné, aby si všichni studenti zkoušeli úlohu předem. Proto by se jim mohl hodit simulátor, který by jednoduše spustili na svém počítači a na kterém by si mohli nastavování síťových parametrů na linuxu a ciscu zkusit. Právě návrhem a implementací takového síťového simulátoru počítačů s OS Linux se zabývá tato bakalářská práce. 

% zkrácenej úvod:
Jednou z laboratorních úloh předmětu Počítačové sítě (Y36PSI) na FELu je postavení sítě mezi několika linuxovými a ciscovými počítači. Studenti, kteří tento úkol plní, nemají často s takovou činností žádnou osobní zkušenost, a tak během úlohy řeší různé banální problémy, kterým by se mohli vyhnout, kdyby měli možnost zkusit si nastavit podobou síť již před samotnou laboratorní úlohou. Mohl by se jim hodit simulátor, který by jednoduše spustili na svém počítači a na kterém by si mohli nastavování síťových parametrů na linuxu a ciscu zkusit. Právě návrhem a implementací takového síťového simulátoru počítačů s OS Linux se zabývá tato bakalářská práce. 


\section{Cíle práce}

Cílem práce je v programovacím jazyce Java SE navrhnout a implementovat aplikaci, která umožní vytvoření virtuální počítačové sítě, pro potřeby předmětu Y36PSI. Z pohledu uživatele by aplikace měla vypadat stejně jako reálná síť. Uživatel spustí aplikaci v konsoli a pak se pomocí telnetu připojí k jejím jednotlivým virtuálním počítačům, podobně jako protokolem ssh k počítačům s OS Linux. Aplikace bude podporovat příkazy potřebné ke konfiguraci síťových rozhraní (\verb|ifconfig|, \verb|ip address|), směrování (\verb|route|, \verb|ip route|) a překladu adres (\verb|iptables -t nat|). Pro ověření správnosti konfigurace sítě budou implementovány příkazy \verb|ping| a \verb|traceroute|.

Nastavenou konfiguraci sítě bude možné uložit do souboru a zase ji ze souboru načíst. Uživatel bude mít možnost vytvářet libovolné sítě s libovolným počtem počítačů typu linux nebo cisco tak, že infrastrukturu sítě napíše do konfiguračního souboru a pak ji z něho načte.


% \section{Rozdělení práce}
% 
% Protože kompletní síťový simulátor pro počítače s Cisco IOS i OS Linux by přesahoval rozsah jedné bakalářské práce, pracovali jsme na ní jako tým s mým kolegou Stanislavem Řehákem. Práce byla rozdělena na 3 části:
% \begin{itemize}
%  \item \textbf{Jádro aplikace}\\ 
% Jedná se především o datové struktury virtuálního počítače, komunikaci s uživatelem po síti a načítání a ukladání souborů. Na této části jsme se podíleli oba, každý z n
%  \item \textbf{Linuxová část}\\
% V této části je potřeba napsat parsery linuxových příkazů a  zjistit, jak se chovají počítače s linuxem v síťové komunikaci a toto chování implementovat. 
%  \item \textbf{Ciscová část}\\
% Touto částí se tato práce nezabývá, zabývá se jí bakalářská práce Simulátor virtuální počítačové sítě Cisco Stanislava Řeháka.
% \end{itemize}


\section{Struktura práce}

Ve druhé kapitole této práce popisuji některé již existující simulátory a zamýšlím se nad jejich využitím pro výukové předměty. Ve třetí kapitole provádím analýzu aplikace. V následujících dvou kapitolách popisuji její realizaci, nejprve realizaci aplikace jako celku a poté realizací jednotlivých příkazů, které jsou v simulátoru implementovány. V závěru práce navrhuji některá vylepšení vytvořeného simulátoru a rozebírám jeho testování.




\chapter{Existující řešení}

Existujících síťových simulátorů je na trhu mnoho, avšak ne všechny splňují podmínky, aby byly jednoduše využitelné pro předmět PSI. Některé nejsou vůbec tvořené pro výukové účely. Nenašel jsem mnoho simulátorů, které by zároveň podporovaly síťové prvky s OS Linux i Cisco IOS a byly vhodné k výukovým účelům.



%------------------------------------------------------------------------------------------------------

\section{Packet tracer}

Tento velmi známý program nemohu v této kapitole vynechat. Je to program přímo od společnosti Cisco, který věrně simuluje různé cisco switche a routery. V simulátoru je jen málo odchylek od skutečných zařízení\cite{wiki:packetTracer}. Má grafické uživatelské rozhraní (obrázek \ref{obr_packet-tracer}), umí i zobrazovat pohyb paketů v síti. Pro potřeby předmětu Y36PSI má však velké nevýhody. Tou první je, že je volně dostupný pouze členům Cisco Networking Academy. Druhá nevýhoda je, že tento simulátor neumožňuje simulovat i počítače s OS Linux.

% packet tracer
\begin{figure}[h]
\begin{center}
\includegraphics[width=12cm]{obrazky/packet-tracer}
\caption{Packet tracer}
\label{obr_packet-tracer}
\end{center}
\end{figure}



%------------------------------------------------------------------------------------------------------

\section{AdventNet Simulation Toolkit}

% AdventNet Simulation Toolkit 6
% AdventNet Simulation Toolkit 6 [7] je kompletný grafický softvérový simulátor
% zariadení a sietí s veľkou podporou viacerých protokolov, Cisco IOS zariadení a iných
% zariadení ako napr. rôzne Linux, či Windows 2000 servery. Tento simulátor je dostupný
% vo verzii nie len pre Windows, ale aj pre Linux a Solaris. Zo stránky sa dá po poskytnutí
% osobných údajov stiahnuť plne funkčná skúšobná verzia na 30 dní. Je síce limitovaná
% na maximálny počet 25 simulovaných prvkov, ale na začlenenie do tejto práce je postačujúca.

\uv{AdventNet Simulation Toolkit je kompletní grafický softwarový simulátor zařízení a sítí s velkou podporou mnohých protokolů, Cisco IOS zařízení a jiných zařízení jako např. různé Linux, či Windows 2000 servery.}\cite{resersni_bakalarka} Argumentem proti použití tohoto programu je především jeho cena, plná časově neomezená verze stojí od \$995 do \$14995 \footnote{k 23.5.2010}. K disposici je zkušební třicetidenní verze.



%------------------------------------------------------------------------------------------------------

\section{Simulační software Omnet++}

\uv{Simulační systém OMNeT++ [11] je velmi propracovaný opensource nástroj pro simulaci prakticky čehokoliv. OMNeT++ je postaven na modulární architektuře, takže při správných knihovnách (modulech) může simulovat počítačovou síť. Systém dokáže simulovat Cisco IOS i počítač postavený na linuxu.}\cite{resersni_bakalarka}. Systém se však hodí spíše pro simulaci zatížení sítě a pro simulaci síťových protokolů. K výukovým účelům není příliš vhodný kvůli své složitosti.

Více se tímto simulátorem zabýval Bc. Jan Michek v~rámci své diplomové práce Emulátor počítačové sítě \cite{reserse:omnet_dp}.

% omnet
\begin{figure}[h]
\begin{center}
\includegraphics[width=12cm]{obrazky/omnet}
\caption{Omnet++}
\label{obr_omnet}
\end{center}
\end{figure}



%------------------------------------------------------------------------------------------------------

\section{Závěr}

Tvůrci simulačních programů dávají většinou přednost simulaci sítí založených na síťových prvcích od firmy Cisco. Podporují-li i počítače s linuxem, ani o tom moc nepíší. Většina simulátorů je studentům nedostupná kvůli své ceně. Studenti předmětu PSI budou simulátor potřebovat pravděpodobně jen několikrát za semestr a je nesmyslné, aby si škola, nebo sami studenti kupovali kvůli několika použitím licenci.



\chapter{Analysa aplikace}

V této kapitole se zabývám analysou a návrhem aplikace jako celku. Shrnuji a analysuji požadavky, diskutuji zvolený jazyk a uživatelské rozhraní, navrhuji architekturu aplikace a odhaduji její náročnost. Analysa jednotlivých částí simulátoru je popisována společně s těmito částmi.


%-----------------------------------------------------------------------------------

\section{Požadavky na aplikaci}

Nejprve shrnu všechny požadavky na mojí aplikaci.

\subsection{Funkční požadavky}
\begin{enumerate}
 \item Vytvoření počítačové sítě založené na počítačích OS Linux.
 \item Aplikace umožňuje konfiguraci rozhraní pomocí příkazů ifconfig a ip addr.
 \item Aplikace obsahuje funkční směrování a umožňuje jeho nastavování pomocí příkazů route a ip route.
 \item Aplikace implementuje překlad adres
 \item Aplikace podporuje ukládání a načítání do/ze souboru.
 \item Pro ověření správnosti jsou implementovány příkazy ping a traceroute.
 \item K jednotlivým počítačům aplikace je možné se připojit pomocí telnetu.
 \item Pomocí telnetu bude možno se připojit zároveň k více virtuálním počítačům.
 \item Pomocí telnetu bude možno připojit se k jednomu počítači vícekrát najednou.
\end{enumerate}

\subsection{Nefunkční požadavky}
\begin{enumerate}
 \item Aplikace bude multiplatformní - alespoň pro operační systémy Windows a Linux
 \item Aplikace musí být spustitelná na běžném\footnote{Slovem \uv{běžné} se myslí v podstatě jakýkoliv počítač, na kterém je možné nainstalovat prostředí Javy - Java Runtime Environment} studentském počítači.
 \item Aplikace by měla být co nejvěrnější kopií reálného počítače s Linuxem.
\end{enumerate}



%-----------------------------------------------------------------------------------

\section{Analysa požadavků}


\subsection{Připojení pomocí telnetu}

Jedním z funkčních požadavků mé aplikace je možnost připojit se k jednotlivým virtuálním počítačům pomocí protokolu telnet. Tento požadavek vypadá jednoduše, pokud pod pojmem Telnet chápeme jednoduchý protokol na přenos textových dat. Takový protokol ovšem neumožňuje doplňování příkazů a jejich historii, což je pro práci s počítačem, byť virtuálním, obrovské omezení. Oproti tomu, implementovat telnet protokol, jako NVT\footnote{NVT – Network Virtual Terminal, česky: Síťový virtuální terminál; poskytuje standardní rozhraní příkazové řádky}, kde se posílá a potvrzuje každý napsaný znak, by překračovalo rozsah této bakalářské práce. Můj kolega nalezl program rlwrap, který poskytuje historii příkazů a jejich doplňování na straně klienta. Funguje v linuxu, na windowsu jen pod cygwinem. Toto druhé řešení bude o dost jednodušší a rozhodli jsme se ho realisovat, i když pro uživatele bude nevýhodou spouštění přes cygwin. I tak ovšem základní požadavek, že s aplikací bude možno komunikovat pomocí telnetu, zůstane zachován, uživatel ovšem přijde o komfort, který mu nabízí možnost doplňování, editace a historie příkazů. 


\subsection{Podobnost simulátoru se skutečným linuxem}

Aby byl simulátor využitelný pro výukové účely, musí být dostatečně podobný skutečnému linuxu, aby uživatel mohl věřit, že to, co funguje v simulátoru, bude fungovat i na skutečném linuxu a naopak. K tomu bude stačit implementovat jen ty příkazy, kterými se nastavují síťové parametry, a jen v takovém rozsahu, jaký je pro tyto výukové účely potřeba. Budu tedy implementovat příkazy \verb|ifconfig\verb|, \verb|route|, \verb|ping| a \verb|traceroute|, z příkazu \verb|ip| stačí implementovat jeho podpříkazy \verb|addr| a \verb|route|. Pro potřeby nastavení překladu adres je potřeba implementovat malou část příkazu \verb|iptables|. Aby uživatel mohl nastavovat některé hodnoty souborů v adresáři \verb|/proc|, implementuji ve velmi omezené míře i příkazy \verb|cat| a \verb|echo|, ovšem jen pro tyto soubory. Pro ukončení spojení bude implementován příkaz \verb|exit|. Pro potřeby simulátoru ale není potřeba implementovat kompletní příkazy \verb|ifconfig| nebo \verb|ip|, ale jen tu jejich část, kterou se nastavují parametry rozhraní, jako IP, maska a další. O ostatních parametrech pak vetšinou simulátor vypíše, že ve skutečnosti sice existují, ale simulátorem zatím nejsou podporované.


\subsection{Počet simulovaných počítačů}

Na laboratořích Y36PSI studenti konfigurují 4 počítače, náš simulátor by měl zvládnout simulovat síť o 10 počítačích. Více ani není potřeba, pro výukové účely studenti pravděpodobně nebudout konfigurovat více počítačů.



%-----------------------------------------------------------------------------------

\section{Programovací jazyk a uživatelské rozhraní}

\subsection{Programovací jazyk}
Aplikaci jsme se rozhodli programovat v programovacím jazyku Java z několika důvodů. Java je programovací jazyk, který nabízí velký programátorský komfort, stabilitu a zároveň možnost vytvořené aplikace používat pod různými operačními systémy, což je další z nefunkčních požadavků. Tento jazyk navíc disponuje hotovými knihovnami pro práci se sítí v balíčku java.net. Dalším důvodem je také to, že s programováním aplikací v Javě mám zatím asi největší zkušenosti.

\subsection{Uživatelské rozhraní}

Jak plyne ze zadání, uživatel se přihlašuje k jednotlivým virtuálním počítačům pomocí programu telnet, nemusím tedy vytvářet žádného speciálního klienta. S aplikací samotnou nebude uživatel nijak pracovat, jenom ji spustí se správným konfiguračním souborem a případně číslem výchozího portu, dále již bude nastavovat pouze jednotlivé virtuální počítače pomocí telnetu. Pro takovou aplikaci je nejlepším uživatelským rozhraním příkazová řádka, vytváření grafického uživatelského rozhraní by nemělo smysl.



%-----------------------------------------------------------------------------------

\section{Návrh architektury}

Aplikace se bude skládat ze dvou vrstev. Komunikační vrstva by měla zajišťovat síťovou komunikaci s klientem, tedy odesílání a přijímání textových dat. Z velké části bude převzata z jiné práce, kterou jsme kdysi dělali jako domácí úkol na předmět Y36PSI. Aplikační vrstva bude tvořena samotnou virtuální sítí. Tyto vrstvy však od sebe nebudou striktně odděleny. Nejprve si rozebereme druhou vrstvu.


\subsection{Virtuální síť}

Virtuální počítačová síť, kterou bude aplikace simulovat, má poskytovat především tyto funkcionality:
\begin{itemize}
 \item Možnost konfigurace jednotlivých síťových prvků.
 \item Posílání paketů mezi síťovými prvky.
\end{itemize}
Skutečná počítačová síť se skládá ze síťových prvků různých druhů. Stejně tak i virtuální síť se bude skládat ze síťových prvků, které budou interně reprezentovány objekty.

\subsubsection{Síťové prvky}

V laboratořích předmětu Y36PSI studenti nastavují pouze PC nebo směrovače na 3. (síťové) vrstvě ISO/OSI modelu\footnote{3. vrstva ISO modelu, tzv. síťová vrstva, zajišťuje spojení mezi jakýmikoliv 2 uzly sítě.}. Síťové prvky pracující na 2. vrstvě ISO/OSI modelu\footnote{2. vrstva ISO/OSI modelu, tzv. spojová nebo linková vrstva, zajišťuje spojení mezi dvěma sousedními systémy.}, switche a bridge se v laboratořích vůbec neuvažují. Proto i ve své práci uvažuji jediný druh síťových prvků - počítače s OS Linux.

\subsubsection{Posílání paketů}
Virtuální síť musí umět posílat virtuální pakety, aby uživatel pomocí příkazů \verb|ping| nebo \verb|traceroute| zjistil, jestli virtuální síť správně nakonfiguroval. Posílání paktů bude vnitřně realisováno vzájemným voláním metod virtuálních počítačů, které si mezi sebou budou předávat objekty typu paket. Tyto metody zřejmě bude vhodné rozdělit tak, aby odpovídaly jednotlivým vrstvám ISO/OSI modelu.


\subsection{Komunikační vrstva}
Komunikační vrstva simulátoru bude zajišťovat spojení aplikace s klientem. Z tohoto pohledu bude simulátor klasickým síťovým serverem, který poslouchá na několika portech, přijímá spojení a zpracovává je. Uživatel bude po síti konfigurovat jednotlivé virtuální počítače, proto každý virtuální počítač musí poslouchat na jednom portu. Pro obsluhu této komunikace bude vytvořeno několik tříd. Aby mohl simulátor poslouchat na více portech najednou, bude nutné vytvořit více vláken, každý virtuální počítač tedy poběží v samostatném vláknu. Jak plyne z posledního funkčního požadavku, musí jeden virtuální počítač umět zpracovat i více spojení najednou, jako i na reálný linuxový počítač je možné se připojit k několika jeho terminálům pomocí protokolu ssh nebo telnet. Proto bude nutné, aby vlákno, které poslouchá na portu, pro příchozí spojení vytvořilo jiné vlákno, které spojení obslouží, a samo dále poslouchalo na určeném portu. 



%-----------------------------------------------------------------------------------

\section{Odhad náročnosti aplikace}

Aplikace nebude mít žádné uživatelské rozhraní, nebude přistupovat do žádné database a její datové struktury budou pravděpodobně poměrně jednoduché. Proto pro virtuální síť o deseti počítačích by simulátor neměl mít žádné zvláštní paměťové nebo procesorové nároky.



%-----------------------------------------------------------------------------------

\section{Odhad složitosti práce a jejího průběhu}


\chapter{Implementace virtuální sítě}

V této kapitole se zabývám analýzou a implementací jednotlivých částí aplikace. Nejdříve popisuji architekturu aplikace jako celku, dále rozebírám analýzu a implementaci třídy \verb|IpAdresa|, implementaci virtuálního počítače, analýzu a implementaci routovací tabulky a analýzu a implementaci posílání paketů. Nezabývám se zde analýzou a implementací jednotlivých příkazů, vzhledem k rozsáhlosti tohoto tématu jsem ho vyčlenil do zvláštní kapitoly, která následuje za touto kapitolou.



%----------------------------------

\section{Popis architektury aplikace}

Aplikace se skládá ze dvou vrstev. První, komunikační vrstva, zajišťuje veškerou síťovou komunikaci s klientem, druhá, aplikační vrstva, reprezentuje virtuální síť, která je simulována. 


\subsection{Komunikační vrstva}\label{impl_komunikacni_vrstva}

Komunikační vrstva byla z velké části přejata z úkolu na předmět Y36PSI, který jsme programovali na podzim roku 2008. Síťová komunikace s klientem je velmi jednoduchá, server s klientem si navzájem posílají jen textová data, tzn. klient posílá serveru příkazy v textové podobě a server na ně odpovídá.

% struktura komunikační vrstvy
\begin{figure}[h]
\begin{center}
\includegraphics[width=14cm]{obrazky/komunikacni_vrstva}
\caption{Komunikační vrstva}
\label{obr_komunikacni_vrstva}
\end{center}
\end{figure}

Hlavní třídou komunikační vrstvy je třída \verb|Komunikace|. Ta se stará o veškerou komunikaci virtuálního počítače s uživatelem. Je potomkem třídy \verb|Thread|. Běží ve vlastním vlákně, které se startuje v jejím konstruktoru, a poslouchá na portu, který jí byl zadán. Pro každé nové příchozí spojení vytvoří instanci třídy \verb|Konsole|, která spojení obslouží, aby \verb|Komunikace| mohla dále poslouchat na portu a zpracovávat další spojení. Třída \verb|Konsole| je také potomkem třídy Thread. Obsluhuje jedno telnetové připojení. Drží si instanci třídy \verb|ParserPrikazu| z balíčku \verb|Prikazy| (o něm v následující kapitole). Přijímá textová data od uživatele až po enter (sekvence \verb|\r\n|), tedy vlastně načítá data po řádcích. Každý řádek, který uživatel pošle, předá parseru na zpracování a pak sama pošle uživateli prompt. Parseru poskytuje metody pro posílání textových dat uživateli. Pro uživatele tak komunikace s touto konsolí vypadá stejně jako práce s příkazovou řádkou na skutečném počítači.


\subsection{Aplikační vrstva - virtuální síť}

Před tím, než začnu popisovat implementaci jednotlivých komponent aplikační vrstvy, sluší se, popsat ji zhruba jako celek. Virtuální síť se skládá z virtuálních počítačů, což jsou objekty potomků abstraktní třídy \verb|AbstraktniPocitac|. Ty si mezi sebou voláním svých metod předávají pakety, což jsou objekty třídy \verb|Paket|. Virtuální počítače mají síťová rozhraní, což jsou objekty třídy \verb|SitoveRozhrani|. Jejich pomocí jsou počítače mezi sebou propojeny (více o infrastruktuře v \ref{infrastruktura_site})

% uml virtualni sit
\begin{figure}[h]
\begin{center}
\includegraphics[width=14cm]{obrazky/virtualni_sit}
\caption{Architektura aplikační vrstvy - virtuální síť}
\label{obr_virtualni_sit}
\end{center}
\end{figure}




%----------------------------------

\section{IP adresa}

Třída \verb|IpAdresa| je sice jen jednou z mnoha tříd, vzhledem k jejímu významu ji ale v následujících odstavcích popíšu podrobněji.


\subsection{Analýza}

Protože simulátor se zabývá především simulací síťové vrstvy ISO/OSI modelu, je síťová adresa počítače, tzv. IP adresa velmi často používanou datovou strukturou, pro kterou se vyplatí mít speciální třídu. Ta se v v aplikaci jmenuje \verb|IpAdresa| a patří do balíčku \verb|datoveStruktury|. Je používána jako parametr síťového rozhraní, jako prefix v routovací tabulce, jako zdrojová a cílová adresa v paketech. Při bližším pohledu je zřejmé, že kontext jejího použití se v těchto případech částečně liší. Například pro posílání paketů je nutné, aby paket obsahoval zdrojovou a cílovou adresu i s portem. Port by samozřejmě nemusel být součástí adresy, to se ale ukázalo jako jednoduší a pro posílání paketů přehlednější možnost. Pro IP adresu jako parametr rozhraní je naopak port zcela nesmyslný parametr, nutně ale potřebuje parametr pro síťovou masku, která je naproti tomu nesmyslná pro posílání paketů. Paket posílám na IP adresu, ne na adresu s maskou.


\subsection{Vnitřní reprezentace}

Přes tyto rozdíly jsem se rozhodl vytvořit pro IP adresu jednu třídu, která má parametry \verb|adresa|, \verb|maska| a \verb|port|, přičemž pro danou situaci nepotřebné parametry prostě ignoruji. Protože Java neobsahuje žádný 32-bitový bezeznaménkový datový typ, jsou parametry adresa a maska vnitřně reprezentovány 32-bitovým integerem, který ale obsahuje bity skutečné adresy, jeho číselná hodnota není důležitá. Operace s nimi se provádí především pomocí bitových operátorů. Parametr port je normální integer.


\subsection{Veřejné metody}

\verb|IpAdresa| má konstruktory, aby ji bylo možné vytvořit ze \verb|Stringu|, s maskou zadanou jako \verb|String|, \verb|Integer| nebo v jednou řetězci s adresou. Adresu je možné převést na String nebo porovnat s jinou adresou mnoha různými způsoby, například jen podle adresy, adresy s portem, adresy s maskou nebo čísla sítě. \verb|IpAdresa| umí vrátit své číslo sítě nebo broadcast jako jinou \verb|IpAdresu|. O těchto metodách se zde nerozepisuji podrobně, v kódu jsou dobře okomentované.



%----------------------------------

\section{Virtuální počítač}

Virtuální počítač je základním stavebním prvkem naší aplikace. Pracuje na obou jejích vrstvách. Na vrstvě komunikační přijímá a zpracovává příchozí spojení, na aplikační vrstvě, tj. na vrstvě virtuální sítě přijímá, posílá a přeposílá pakety. Posíláním paketů se zabývám až v posledním odstavci této kapitoly, zde proberu komunikační vrstvu počítače a jeho rozhraní.

Protože linuxový a ciscový počítač, který dělal kolega, mají mnoho společného, vytvořil jsem abstraktní třídu \verb|AbstraktniPocitac|, který je předkem počítačů obou typů. Třída \verb|LinuxPocitac| má ale jen jednu metodu, která se týká posílání paketů, proto se jí zatím nezabývám.

Všechny virtuální počítače jsou vytvářeny v rámci inicializace aplikace dle konfiguračního souboru na začátku jejího běhu, za chodu aplikace není již možné další počítač přidat nebo nějaký odebrat. Pro komunikaci s uživatelem má každý počítač vlastní objekt třídy \verb|Komunikace|. Počítač si drží seznam svých síťových rozhraní, svoji routovací tabulku a natovací tabulku. Má jediný konstruktor, kde je mu zadáno jméno (pro přehlednost) a port, na kterém má být dostupný pro uživatele.


\subsection{Síťové rozhraní}

%v tomhle odstavci by chtěla dodělat ta footnote
Z hlediska infrastruktury sítě jsou základními prvky počítače jeho síťová rozhraní. Ty si počítač drží v seznamu. Jsou vytvořeny při parsovaní konfiguračního souboru a během běhu aplikace je nelze nijak měnit, přidávat nebo mazat. Třída \verb|SitoveRozhrani| má svoje jméno a fysickou (mac) adresu. Protože v naší aplikaci není implementován ARP\footnote{Address Resolution Protocol se v počítačových sítích s IP protokolem používá k získání ethernetové MAC adresy sousedního stroje z jeho IP adresy. Používá se v situaci, kdy je třeba odeslat IP datagram na adresu ležící ve stejné podsíti jako odesílatel. Data se tedy mají poslat přímo adresátovi, u něhož však odesilatel zná pouze IP adresu. Pro odeslání prostřednictvím např. Ethernetu ale potřebuje znát cílovou ethernetovou adresu.\cite{wiki:arp}} protokol, mac adresa nemá jiný význam, než že je vypisována příkazy jako např. \verb|ifconfig|.

Skutečné síťové rozhraní může mít více adres. Tato možnost však není v předmětu PSI využívána, proto jsem ji neimplementoval. Znamenalo by to totiž poměrně velké problémy v posílání paketů. Musel bych složitě zjišťovat, kdy se paket odešle s jakou síťovou adresou, pokud je jich na daném rozhraní více. Pro potřeby statického překladu adres (NAT) především na ciscovém routeru je ale nutné mít na rozhraní více adres\footnote{Více o překladu adres v bakalářské práci mého kolegy Stanislava Řeháka}. Proto má třída \verb|SitoveRozhrani| seznam IP adres, ale jeho první adresa je privilegovaná. Každý paket, který je přes dané rozhraní posílán, má jako odchozí adresu první adresu tohoto rozhraní. První adresa je vždy nastavená, není-li nakonfigurována, je nastavena na \verb|null|. Tuto jedinou adresu lze nastavovat a vypisovat. Ostatní adresy jsou přidávány jen pro potřeby statického natování. Pokud rozhraní nemá nastavenou žádnou adresu, je první (privilegovaná) adresa \verb|null|.



%----------------------------------

\section{Infrastruktura virtuální sítě}\label{infrastruktura_site}

Poté, co jsem popsal implementaci virtuálního počítače, můžu popsat vnitřní reprezentaci infrastruktury virtuální sítě. Ta vychází z toho, že v simulátoru neuvažuji směrovače na linkové vrstvě ISO/OSI modelu (switche). To totiž znamená, že jedno síťové rozhraní počítače může být připojeno nejvýše k jednomu jinému síťovému rozhraní nějakého počítače. Infrastruktura takové sítě je tak jednoznačně určena dvojicemi síťových rozhraní, které jsou mezi sebou propojeny kabelem. Třída \verb|SitoveRozhrani| má proto parametr \verb|pripojenoK|, který obsahuje odkaz na jiné síťové rozhraní, ke kterému je připojeno. Není-li rozhraní připojeno, je tento parametr nastaven na \verb|null|. Tato infrastruktura je samozřejmě načítána z konfiguračního souboru. Při vytváření infrastruktury aplikace ohlídá, aby rozhraní byla spojena obousměrně a správně. To jest, je-li rozhraní \verb|A| připojeno k rozhraní \verb|B|, musí být také rozhraní \verb|B| připojeno k rozhraní \verb|A|.



%----------------------------------

\section{Routovací tabulka}

% OSNOVA
% - co to je, struktura týhle části
% - použití pro cisco
% Analysa
% 	Struktura tabulky
% 	Adresát - vysvětlit, jak to funguje
% 	Příznaky
% 	Přidávání záznamů a jejich řazení
% 	Mazání záznamů
% 	Použití při směrování
% Implementace
% 	Vnitřní reprezentace (struktura)
% 	Přidávání, mazaní a řazení záznamů - jednotlivý metody, metoda pro parser konfiguráku
% 	Použití při směrování - metody

Počítače směrují pakety podle tzv. routovací, neboli směrovací, tabulky. \uv{Routovací tabulka je datový soubor uložený v RAM paměti, který je používán k uchovávání informací ohledně přímo připojených i vzdáleně připojených sítí. Její obsah napovídá routeru, kterým rozhraním je možno nejoptimálněji dosáhnout cílové sítě.}\cite{owebu:routovaci_tabulka}. V této části se zabývám nejprve analýzou routovací tabulky na skutečném linuxu a potom popisuji její implementaci v simulátoru.

Třída \verb|RoutovaciTabulka| měla být původně stejně použitelná pro linux i pro cisco. Až po tom, co jsem jí implementoval, kolega zjistil, že pro potřeby cisca není tato třída bez úprav použitelná. Proto implementoval třídu \verb|CiscoWrapper|, která obaluje třídu \verb|RoutovaciTabulka| a dodává jí funkce potřebné pro cisco. To však není obsahem mojí práce.


\subsection{Analýza routovací tabulky na skutečném počítači}

\subsubsection{Struktura tabulky}

V řádcích routovací tabulky jsou záznamy pro jednotlivé sítě. Každý záznam má tyto parametry:
\begin{itemize}
\item adresát - IP adresa s maskou, pro kterou je tento záznam platný
\item brána - IP adresa počítače, na který se má paket poslat. Tento sloupec nemusí být vždy vyplněn.
\item příznaky - O těch více píšu v samostatné části.
\item metrika - Jedno z kriterií priority.
\item rozhraní - Rozhraní, přes které se paket posílá.
\end{itemize}
Parametr metrika není pro výukové účely potřeba, proto se jím již dále nezabývám.

Pro lepší představu zde vkládám routovací tabulku tak, jak je vypsána příkazem\linebreak \verb|route -n|:
\begin{verbatim}
Adresát         Brána           Maska           Přízn Metrik Rozhraní
147.32.125.128  0.0.0.0         255.255.255.128 U     1      eth0
169.254.0.0     0.0.0.0         255.255.0.0     U     1000   eth0
0.0.0.0         147.32.125.129  0.0.0.0         UG    0      eth0
\end{verbatim}

\subsubsection{Adresát}

V hořejším výpisu tabulky pomocí příkazu \verb|route| se adresáta týkají 2 sloupce, sloupec Adresát a sloupec Maska, ve kterém jsou vypsány masky k IP adresám uvedeným ve sloupci Adresát. Tyto IP adresy s maskami jsou vždy číslem sítě a reprezentují všechny adresy, které do této sítě patří. Tak například adresát 0.0.0.0/0 reprezentuje úplně všechny IP adresy, adresát 147.32.125.128/25 reprezentuje adresy v rozmezí 147.32.125.128 až 147.32.125.255, adresát 1.1.1.1/32 reprezentuje jedinou adresu 1.1.1.1 a adresát 192.168.1.0/24 reprezentuje všechny adresy, které začínají byty 192.168.1.x. Adresáta 147.32.125.128/24 nelze zadat, protože číslo této sítě je 147.32.125.0/24.

\subsubsection{Příznaky}\label{routTabulka-priznaky}

Záznam routovací tabulky má několik příznaků. Má vždy minimálně jeden příznak, může mít ale všechny 3 příznaky najednou. Zde je jejich popis:
\begin{itemize}
\item Příznak \verb|U| znamená, že záznam obsahuje rozhraní. Protože záznam bez vyplněného rozhraní není možné zadat, má tento příznak každý záznam.
\item Záznam má příznak \verb|G|, jestliže je vyplněn sloupec brána.
\item Příznak H znamená, že adresátem daného záznamu je jeden počítač, tzn. adresát má masku 255.255.255.255.
\end{itemize}
Příznak \verb|H| jen informuje, že adresát není sítí, ale jediným počítačem, není tedy nijak důležitý. Podle příznaků existují 2 typy záznamů, záznamy s příznakem \verb|U| a záznamy s příznakem \verb|UG|. Tyto typy se liší jak při přidávání nových záznamů do routovací tabulky, tak při posílání paketu podle tohoto záznamu. Posílá-li počítač paket podle záznamu U, není z tohoto záznamu zřejmé, jakému sousednímu počítači (na linkové vrstvě) se má paket poslat. Počítač se tedy pokusí poslat paket přímo na cílovou IP adresu uvedenou v paketu. Posílá-li se paket podle záznamu UG, posílá se na adresu brány uvedenou v záznamu. Více se této problematice věnuji v kapitole o posílání paketů.

\subsubsection{Přidávání záznamů a jejich řazení}

Routovací tabulka nesmí obsahovat 2 stejné záznamy. Za stejné záznamy se považují záznamy, které mají stejného adresáta včetně masky, stejné rozhraní a stejnou bránu.

Záznam typu \verb|U| lze přidat vždycky. Záznam typu \verb|UG| lze přidat jen pod podmínkou, že jeho brána je v okamžiku přidání dosažitelná záznamem typu \verb|U|. Tím je možné dosáhnout zajímavého chování: Když do routovací tabulky přidám defaultní routu \footnote{Defaultní routa je záznam platný pro celý internet, jeho adresátem je 0.0.0.0/0} záznamu typu U, můžu pak přidat routu na jakoukoliv síť v internetu se záznamem UG. Když potom smažu původní defaultní routu, můžu posílat pakety pouze na tu síť s příznakem \verb|UG| a na počítač v mé síti paket neodešlu. Zde uvádím příklad:
\begin{verbatim}
root: /home/neiss# route add default eth0
root: /home/neiss# route add -net 89.190.94.0/24 gw 89.190.94.1
root: /home/neiss# route del default
root: /home/neiss# route
Směrovací tabulka v jádru pro IP
Adresát         Brána           Maska           Přízn Metrik Rozhraní
89.190.94.0     89.190.94.1     255.255.255.0   UG    0      eth0
\end{verbatim}
Toto je ping na nějakou adresu kdesi v internetu:
\begin{verbatim}
root: /home/neiss# ping -c1 89.190.94.58
PING 89.190.94.58 (89.190.94.58) 56(84) bytes of data.
64 bytes from 89.190.94.58: icmp_seq=1 ttl=53 time=14.1 ms

--- 89.190.94.58 ping statistics ---
1 packets transmitted, 1 received, 0% packet loss, time 0ms
rtt min/avg/max/mdev = 14.141/14.141/14.141/0.000 ms
\end{verbatim}
Toto je ping na mojí bránu, přes kterou do internetu šel i minulý paket:
\begin{verbatim}
root: /home/neiss# ping -c1 147.32.125.129
connect: Network is unreachable
\end{verbatim}
Tento pokus funguje ale jen tehdy, pokud moje brána, v tomto případě 147.32.125.129 je cisco (viz část o posílání paketů).

Záznamy se v tabulce řadí podle masky adresáta. Nahoře jsou záznamy s nejdelší maskou, tzn. záznamy nejkonkrétnější. Pokud vkládám více záznamů se stejnou maskou, chová se routovací tabulka naprosto nepředvídatelně, což je vidět na následujícím příkladě:
\begin{small}
\begin{verbatim}
node-4:/home/dsn# route add -net 1.1.8.0/25 dev eth0
node-4:/home/dsn# route add -net 1.1.9.0/25 dev eth0
node-4:/home/dsn# route add -net 1.1.10.0/25 dev eth0
node-4:/home/dsn# route add -net 1.1.11.0/25 dev eth0
node-4:/home/dsn# route add -net 1.1.12.0/25 dev eth0
node-4:/home/dsn# route add -net 1.1.13.0/25 dev eth0
node-4:/home/dsn# route add -net 1.1.14.0/25 dev eth0
node-4:/home/dsn# route add -net 1.1.15.0/25 dev eth0
node-4:/home/dsn# route
Kernel IP routing table
Destination     Gateway         Genmask         Flags Metric  Iface
1.1.10.0        *               255.255.255.128 U     0       eth0
1.1.11.0        *               255.255.255.128 U     0       eth0
1.1.8.0         *               255.255.255.128 U     0       eth0
1.1.9.0         *               255.255.255.128 U     0       eth0
1.1.14.0        *               255.255.255.128 U     0       eth0
1.1.15.0        *               255.255.255.128 U     0       eth0
1.1.12.0        *               255.255.255.128 U     0       eth0
1.1.13.0        *               255.255.255.128 U     0       eth0
\end{verbatim}
\end{small}
Podle jakého algoritmu jsou nové záznamy zařazovány je mi opravdu záhadou.

\subsubsection{Mazání záznamů}

Smazat je možno jakýkoliv záznam tabulky. Pro mazání záznamů je potřeba zadat správně minimálně adresáta záznamu. Pokud pak existuje více záznamů se zadanými parametry, smaže se první z nich. Smazání jakéhokoliv záznamu nijak neovlivní ostatní záznamy.

\subsubsection{Použití pro směrování}\label{routTabulka_pouzitiPriSmerovani}

Ke směrování paketu se použije první záznam odpovídající cílové adrese. To znamená, že bude-li v routovací tabulce dané adrese odpovídat více záznamů, použije se ten nejvíce nahoře. Protože záznamy jsou řazeny podle délky síťové masky, je vrácený záznam ten nejkonkrétnější.


\subsection{Implementace routovací tabulky v simulátoru}

Routovací tabulka je implementována třídou \verb|RoutovaciTabulka|, jejíž odkaz si drží\linebreak \verb|AbstraktniPocitac| a podle ní směruje pakety.

\subsubsection{Vnitřní reprezentace}

Tabulka je vnitřně reprezentována seznamem objektů typu \verb|Zaznam|, který reprezentuje jeden záznam, tj. řádek tabulky. Záznam routovací tabulky má v simulátoru jen tyto parametry: adresát, brána a rozhraní, které fungují tak, jak bylo popsáno v odstavci o analýze. Parametry adresát a brána jsou typu \verb|IpAdresa|, parametr rozhraní je typu \verb|SitoveRozhrani|. Parametr záznamy není vůbec potřeba. Příznak \verb|U| musí mít záznam vždy, příznak \verb|H| má právě tehdy, když adresát má masku 255.255.255.255, a příznak \verb|U| má záznam právě tehdy, když má vyplněnou položku brána, proto ani ten není potřeba.

\subsubsection{Přidávání, mazání a řazení záznamů}

Záznamy jsou přidávány pomocí 2 metod se stejným názvem \verb|pridejZaznam|, ale jinými parametry. Jedna přidává záznam typu \verb|U|, druhá, která má navíc parametr brána, záznam typu \verb|UG|. U obou se kontroluje, jestli tabulka již stejný záznam neobsahuje, u té druhé se navíc kontroluje dosažitelnost brány, jak bylo popsáno v analýze.

Záznamy se samozřejmě řadí podle masky jako v reálné tabulce, záznamy se stejnou maskou se ale vloží vždy nad původní záznam. Tak jsou novější záznamy vždy nahoře. Zmatečné řazení reálné tabulky jsem samozřejmě neimplementoval.

Pro mazání má \verb|RoutovaciTabulka| metodu \verb|SmazZaznam|, funguje stejně jako na reálném počítači.

Navíc obsahuje \verb|RoutovaciTabulka| ještě metodu \verb|pridejZaznamBezKontrol|, která je využívána při vytváření počítače z konfiguračního souboru, jinde se nepoužívá.

\subsubsection{Použití při směrování}

K samotnému směrování slouží metoda \verb|najdiSpravnyZaznam|, která vrací celý řádek routovací tabulky. Funguje stejně jako na reálném počítači.



%----------------------------------

\section{Posílání paketů}

% 	Osnova sekce:
% 	- popsat, k čemu se u nás pakety posílaj
% 	- popsat strukturu týhle části - DODĚLAT !!!
% 	Referenční model ISO/OSI
% 		- popsat vrstvy skutečný sítě, který se nás týkaj, zabalování rámců, atd
% 		Datové bloky
% 	implementace našeho paketu
% 	Chování reálného počítače při posílání paketů
% 		Trensportní vrstva - icmp
% 		Síťová vrstava - IP
% 			ip_forward
% 			Směrování
% 			ttl
% 			Next hop
% 		Linková vrstva - ethernet
% 		Zajímavosti - DODĚLAT!!!
% 	popsat implementaci
% 		- rozdělení do vrstev
% 		jednotlivý metody

Posílání paketů v naší aplikaci slouží k tomu, aby uživatel pomocí příkazů \verb|ping| a \verb|traceroute| mohl ověřit, zda síť správně nakonfiguroval. Bez této části by naší aplikaci nebylo možné nazvat síťovým simulátorem.


\subsection{Teoretický rozbor referenčního modelu ISO/OSI}

\uv{Referenční model ISO/OSI vypracovala organizace ISO jako hlavní část snahy o standardizaci počítačových sítí.}\cite{wiki:referencni_model}. Dle tohoto modelu probíhá síťová komunikace v sedmi vrstvách, z níž každá poskytuje přesně definované funkce a komunikuje jen s vrstvou sousední. Každá vrstva má svůj formát přenášených dat, obyčejně dělených do bloků. Pro moji aplikaci jsou důležité vrstvy 2 - 4, tzn. spojová, síťová a transportní vrstva.

\subsubsection{Spojová vrstva}

Spojová nebo linková vrstva\footnote{Více v \cite{wiki:linkova_vrstva}} \uv{poskytuje spojení mezi dvěma sousedními systémy.}\cite{wiki:referencni_model}. Tuto vrstvu zajišťuje na skutečné síti v laboratoři technologie Ethernet\cite{wiki:ethernet}, ke zjištění fysické adresy sousedního systému se používá protokol ARP\cite{wiki:arp}. Sousedními systémy se zde rozumí počítače zapojené do stejné sítě. Blok dat na linkové vrstvě se nazývá rámec. Vzhledem k tomu, že simulátor neobsahuje žádné switche, zajišťuje tato vrstva v naší aplikaci spojení mezi 2 počítači propojenými kabelem.

\subsubsection{Síťová vrstva}

Síťová vrstva \uv{se stará o směrování v síti a síťové adresování. Poskytuje spojení mezi systémy, které spolu přímo nesousedí.}\cite{wiki:referencni_model} Zajišťuje spojení mezi jakýmikoliv dvěma uzly sítě. Obyčejně je realizována protokolem IP\footnote{Internet Protocol}. Blok dat na síťové vrstvě se nazývá paket.

\subsubsection{Transportní vrstva}

Transportní vrstva \uv{zajišťuje přenos dat mezi koncovými uzly}\cite{wiki:referencni_model}. Pro potřeby příkazů ping a traceroute je tato vrstva realisována protokolem ICMP\footnote{Internet Control Message Protocol, více v \cite{wiki:icmp}}. Blok dat se v transportní vrstvě nazývá datagram.

\subsubsection{Datové bloky}\label{datove_bloky}

Datový blok jakékoliv vrstvy se skládá z hlavičky, která obsahuje režijní informace té vrstvy a z datové části, která obsahuje samotná data, ale i hlavičky vyšších vrstev. Tak například datagram protokolu ICMP je obalen hlavičkou IP na síťové vrstvě a hlavičkou protokolu Ethernet na spojové vrstvě.


\subsection{Implementace třídy Paket}
Pro můj síťový simulátor by bylo nesmyslné implementovat posílání paketů včetně jejich zabalování do datových bloků různých vrstev, tak jak je to popsáno v předchozím odstavci \ref{datove_bloky}. Proto jsem vytvořil třídu \verb|Paket|, která obsahuje všechny potřebné informace z datových bloků všech tří vrstev. \verb|Paket| má parametry síťové vrstvy jako zdrojovou a cílovou adresu \verb|ttl| a parametry transportní vrstvy jako typ a kód protokolu ICMP. Pro snadnější implementaci příkazu ping má parametr \verb|cas|, kam se ukládá náhodně generovaný čas běhu paketu, který vypisuje příkaz ping. Protože na jednom virtuálním počítači může běžet více příkazů \verb|ping| najednou, nese paket i odkaz na příkaz, který ho poslal, aby ho tento příkaz mohl také po jeho návratu zpracovat.

Typy a kódy ICMP paketů jsou označeny stejně jako ve skutečnosti:
\begin{itemize}
\item typ 0 - ozvěna (icmp reply) - odpověď na požadavek icmp request
\item typ 3 - vyslaný paket nemohl být doručen
\item typ 8 - žádost o ozvěnu (icmp request)
\begin{itemize}
\item kód 0 - network unreachable (nedosažitelná síť)
\item kód 1 - host unreachable (nedosažitelná adresa)
\end{itemize}
\item typ 11 - \verb|ttl| vypršelo
\end{itemize}


\subsection{Chování reálného počítače při posílání paketů}

\subsubsection{Transportní vrstva - protokol ICMP}

V souboru \verb|/proc/sys/net/ipv4/icmp_echo_ignore_all| je nastaveno, jestli počítač odpovídá na dotazy icmp reply. Pokud je v tomto souboru \verb|0|, počítač na dotazy odpovídá. Toto je defaultní nastavení, proto jsem v simulátoru tento problém vůbec neřešil a počítač odpovídá na icmp request vždy.

\subsubsection{Síťová vrstva - protokol IP}

\paragraph{Přeposílání paketů}
Počítač přeposílá pakety jenom tehdy, pokud je v souboru \verb|/proc/sys/net/ipv4/ip_forward| jednička, jinak ne. Toto nastavení ale již není defaultní, proto ho musím v simulátoru implementovat. Proměnná \verb|ip_forward| je parametrem virtuálního počítače a je nastavována pomocí příkazu \verb|echo|.
\paragraph{Směrování paketů}
Pakety jsou směrovány podle routovací tabulky, kde se vybere první záznam odpovídající cílové adrese paketu, jak je popsáno v \ref{routTabulka_pouzitiPriSmerovani}. Pokud v routovací tabulce nebyl nalezen žádný záznam pro cílovou adresu paketu, pošle se na jeho zdrojovou adresu paket \verb|icmp_net_unreachable|\footnote{Tj ICMP paket typu 3 kódu 0}.
\paragraph{ttl}
Každý prvek, který pracuje i na síťové vrstvě sníží procházejícím paketům hodnotu \verb|ttl|. Pokud po tomto snížení dosáhne hodnota \verb|ttl| nuly, pošle počítač na zdrojovou adresu paketu zprávu \verb|icmp_ttl_exceeded|\footnote{Tj. icmp paket typu 11}.
\paragraph{Next hop}
Předtím než síťová vrstva předá odeslání paketu linkové vrstvě, musí pro tento paket zjistit tzv. next hop, neboli sousední adresu. \uv{Next hop je sousední směrovač, na který je paket poslán nebo přeposlán z daného směrovače na své cestě k cíli.}\cite{next_hop} Tato adresa je velmi důležitá pro linkovou vrstvu, která paket posílá právě na tuto adresu. Pokud je paket směrován podle záznamu typu \verb|UG| (viz \ref{routTabulka-priznaky}), je adresa next hop uvedena ve sloupci brána routovací tabulky. Pokud je paket směrován podle záznamu typu \verb|U|, je adresa next hop cílová adresa paketu. Podle routovací tabulky
\begin{verbatim}
Adresát         Brána           Maska           Přízn Metrika  Rozhraní
147.32.125.128  0.0.0.0         255.255.255.128 U     0       eth0
0.0.0.0         147.32.125.129  0.0.0.0         UG    0       eth0
\end{verbatim}
je pro pakety směrované podle prvního záznamu next hop rovná jejich cílové adrese. Pro pakety směrované podle druhého záznamu je next hop \verb|147.32.125.129|. Záznamy typu \verb|U| jsou tak používány pro počítače v mé síti, které jsou dosažitelné přímo bez jakéhokoliv mezilehlého směrovače. Záznamy \verb|UG| jsou používány pro počítače, na které je paket posílán přes jeden nebo více směrovačů.

\subsubsection{Linková vrstva}\label{skutecna_linkova_vrstva}

Na linkové vrstvě se rámce přeposílají jen mezi sousedními počítači. Běžně ji zajišťuje protokol Ethernet. IP adresu sousedního počítače, na kterou má rámec poslat, next hop, dostane od síťové vrstvy, musí ji přeložit na fysickou (MAC\footnote{Media Access Control}) adresu, což dělá protokolem ARP. Protokol ARP vyšle ethernetový rámec na všechny okolní počítače žádost, která obsahuje zadanou IP adresu. Pokud některý z počítačů má tokovou IP adresu, původnímu počítači pošle zpátky svoji fysickou adresu a ten pak může odeslat paket. Aby počítač nemusel fysickou adresu zjišťovat při odesílání každého rámce, ukládá si v datové struktuře nazvané ARP tabulka záznamy s IP adresami, které již dříve překládal.

Pokud počítač nemůže linkovou vrstvou odeslat rámec (paket), například proto, že počítač s takovou adresou na síti neexistuje, pošle původnímu odesílateli ICMP paket\linebreak \verb|net unreachable|\footnote{Tj. ICMP paket typu 3 kódu 1}.

% ethernetový problémy na ciscu:
Počítač s operačním systémem linux odpovídá na ARP dotazy jen tehdy, když má požadovanou IP adresu nastavenou na svém rozhraní. V tom se liší od cisca, které na ARP dotaz odpoví i v případě, že adresu na svém rozhraní nemá, ale ví, kam má paket dále směrovat, tzn. má pro požadovanou adresu záznam v routovací tabulce, a naopak na ARP dotaz neodpoví v případě, že nemá v routovací tabulce záznam pro IP adresu, odkud dotaz přišel. Pro lepší pochopení přepisuji podmínku ještě v jazyce booleovských výrazů:\\
Cisco odpoví na ARP dotaz právě tehdy, když:\\
Má záznam v routovací tabulce pro počítač, který ARP dotaz posílá \verb|a zároveň| (Má nastavenou požadovanou IP adresu \verb|nebo| Má záznam v routovací tabulce pro cílovou adresu posílaného paketu)

\subsubsection{Zajímavá zjištění}

Při analýze chování linuxového počítače v síťové komunikaci jsem zjistil několik zajímavých a aspoň pro mě překvapujících faktů. Například 2 počítače v síti spolu můžou komunikovat i tehdy, mají-li nastaveny IP adresy z úplně jiných sítí. Stačí totiž, mají-li v routovací tabulce jeden na druhého záznam. Počítače z obrázku \ref{obr_divna_sit} spolu opravdu komunikují, jak je vidět na následujícím výpisu.

% divná síť
\begin{figure}[h]
\begin{center}
\includegraphics[width=12cm]{obrazky/divna_sit}
\caption{Funkční síť se \uv{špatnými} adresami}
\label{obr_divna_sit}
\end{center}
\end{figure}

\begin{verbatim}
node-1:/home/dsn# ifconfig eth0
eth0      Link encap:Ethernet  HWaddr 00:40:F4:B7:F0:32
          inet addr:192.168.1.1  Bcast:192.168.1.255  Mask:255.255.255.0
          inet6 addr: fe80::240:f4ff:feb7:f032/64 Scope:Link
          UP BROADCAST RUNNING MULTICAST  MTU:1500  Metric:1
          RX packets:43 errors:0 dropped:0 overruns:0 frame:0

node-1:/home/dsn# ping 10.0.0.1
PING 10.0.0.1 (10.0.0.1) 56(84) bytes of data.
64 bytes from 10.0.0.1: icmp_seq=1 ttl=64 time=0.942 ms
64 bytes from 10.0.0.1: icmp_seq=2 ttl=64 time=0.208 ms

--- 10.0.0.1 ping statistics ---
2 packets transmitted, 2 received, 0% packet loss, time 1003ms
rtt min/avg/max/mdev = 0.191/0.447/0.942/0.350 ms

\end{verbatim}


\subsection{Implementace v simulátoru}

Pakety jsou mezi počítači posílány vzájemným voláním metod, které si mezi sebou předávají objekt typu \verb|Paket|. Všechny tyto metody jsou ve třídě \verb|AbstraktniPocitac|, protože to jsou právě virtuální počítače, které si mezi sebou pakety posílají. Jen metoda \verb|prijmiEthernetove| je sice v \verb|AbstraktniPocitac| deklarována, ale implementována v jeho potomcích.


Metody virtuálního počítače pro posílání paketů jsem bylo rozumné implementovat dle vrstev. Je to přehlednější, než kdybych měl jednu metodu pro přijmutí paketu a je to dobré i vzhledem k tomu, že linux a cisco se v některých případech liší a to jen na některých vrstvách. Metody jedné vrstvy tak volají jen metody vrstvy sousední, jako na reálné síti jedna vrstva komunikuje jen s vrstvami sousedními.

\subsubsection{Linková vrstva}

ARP protokol na zjišťování fysických adres nemusel být implementován, protože v síti nejsou žádné switche a v linkové vrstvě tak není potřeba žádné směrování. Aby ale byly splněny podmínky doručení nebo nedoručení paketu uvedené v \ref{skutecna_linkova_vrstva}, byly vytvořeny metody \verb|odesliEthernetove| a \verb|prijmiEthernetove|, které si mezi sebou předávají pakety tak, aby byly tyto podmínky splněny. Metoda \verb|odesliEthernetove| je volána nějakou metodou síťové vrstvy. Pokouší se odeslat paket tak, že zavolá metodu \verb|prijmiEthernetove| nějakého jiného počítače. Metoda \verb|prijmiEthernetove| na linuxu přijme paket jen tehdy, pokud souhlasí očekávaná adresa, tj. adresa next hop odesílacího počítače. Přijme tedy paket právě tehdy, když by skutečný počítač odpověděl na ARP dotaz. Pokud se metodě \verb|odesliEthernetove| nepovede paket odeslat proto, že ho metoda \verb|prijmiEthernetove| odmítla, pošle odesílateli pomocí metody \verb|posliNovejPaketOdpoved| zprávu \verb|icmp host unreachable|. Pokud metoda \verb|prijmiEthernetove| paket přijme, zavolá metodu \verb|prijmiPaket| síťové vrstvy.

\subsubsection{Síťová vrstva}

Na síťové vrstvě si pakety předávají metody \verb|odesliNovejPaket|, \verb|preposliPaket| a\linebreak \verb|prijmiPaket|. Tyto metody směrují pakety dle routovací tabulky a provádí překlad adres podle natovací tabulky. Všechny pakety přijímá metoda \verb|prijmiPaket|, která se nejdříve pokusí přeložit cílové adresy paketů. Pak rozhodne, je-li přijatý paket na tom počítači v cíli, nebo jestli se má dále přeposlat a případně zavolá metodu \verb|preposliPaket| nebo paket nějak zpracuje, například odpoví na \verb|icmp request|. Metoda \verb|preposliPaket| přeposílá pakety, když má nastaveno \verb|ip_forward|, a přeposílaným paketům snižuje ttl. Při odeslání se pokouší přeložit zdrojovou adresu paketu. Metoda \verb|odesliNovyPaket| slouží k odesílání všech nových paketů vytvořených na tom počítači.

\subsubsection{Transportní vrstva}

Do transportní vrstvy patří více metod, které slouží k posílání různých typů ICMP paketů. Například \verb|posliIcmpRequest|, \verb|posliNetUnreachable| a další. Všechny tyto metody volají metodu \verb|odesliNovyPaket| ze síťové vrstvy.
  
  


\chapter{Implementace příkazů}

V této kapitole popisuji analýzu a implmentaci linuxových příkazů, které jsou v simulátoru implementovány. 

\chapter{Uživatelské testování}

Uživatelské testování prováděl tester za přítomnosti mé a mého kolegy z~týmu. Jeho hlavním účelem bylo najít chyby v~programu, na které jsem sám nepřišel. V~této kapitole popisuji pouze testování těch částí systému, které jsem sám programoval. Testování částí, které dělal kolega, zde nepopisuji.



%----------------------------------------------------------------------------------------------------------

% uziv. zkusil zmenit MAC adresu - chyba
% ifconfig --help vypsal chybu, at uzivatel zkusi ifconfig --help
% ping -c 4 192.168.1.1 - chyba -mezera mezi 'c' a 4
% ping IP prepinace - nefunguje
% man route nefunguje
% route -h a --help a --version nefunguje
% prepsat vypisy z Cj do Aj 
% echo 1 > ip_forward - ma tam byt jen /proc/sys/net/..


\section{Průběh testování}


\subsection{Spuštění aplikace}

Prvním úkolem testera bylo aplikaci spustit a zjistit, jak se s~ní zachází. Toto činilo uživateli menší problémy, protože chybová hlášení vyhazovaná simulátorem nebyla dost jasná na to, aby uživatel pochopil, jak se má chyb vyvarovat. Spouštění aplikace je práce mého kolegy a ten chyby opravil.


\subsection{Práce s~aplikací}

Tester měl za úkol zkonfigurovat síť z obrázku \ref{obr_testovani}. Během konfigurace přišel na chyby, které zde popisuji, a zároveň také popisuji, jak byly opraveny.

% testovací síť
\begin{figure}[h]
\begin{center}
\includegraphics[width=5cm]{obrazky/testovani}
\caption{Testovací síť, kterou tester konfiguroval}
\label{obr_testovani}
\end{center}
\end{figure}

\subsubsection{Výpisy}

Hlášení o~běhu simulátoru, která jsou vypisována na standartní výstup, se testerovi zdála velmi nepřehledná. Výpisy o~tom, co server posílá jednotlivým klientům a co klienti posílají serveru, jsou pro uživatele nadbytečné a proto byly zrušeny. Ponechány byly jen výpisy o~přihlášení klienta a o~průchodu paketu, které byly testerovi užitečné pro diagnostiku sítě.

\subsubsection{Příkaz ifconfig}

Tester zkoušel změnit mac adresu rozhraní, což simulátor nepodporuje. Přidal jsem příkaz \verb|help|, který popisuje, jaké varianty příkazů jsou podporovány.

Tester zkoušel příkaz \verb|ifconfig --help|, který nefunguje. Chybu jsem opravil.

\subsubsection{Manuálové stránky}

Tester zkusil zadat \verb|man route|, přičemž simulátor vypsal, že příkaz neexistuje. Dodělal jsem příkaz \verb|man|, který vypíše, že manuálové stránky nejsou implementovány a doporučí uživateli příkaz \verb|help|.

\subsubsection{Příkaz ping}

Uživatel zadal příkaz \verb|ping 192.168.1.1 -c5|. Zjistil, že parametr \verb|-c| je ignorován, protože parser parsuje jen parametry před adresou. Parser jsem z důvodu velké časové tísně zatím nestihl opravit. Chyby je zmíněna přímo u popisu tohoto příkazu a bude opravena.

\subsubsection{Soubor ipforward}

Z důvodů snažšího testování jsem měl na soubor \verb|/proc/sys/net/ipv4/ip_forward| nastaven alias \verb|ip_forward|, který jsem zapomněl odstranit. Alias již byl odstraněn.

\subsubsection{Příkaz route}

Tester zkoušel příkaz \verb|route --help|, který nefunguje. Do parseru jsem přidal parsování tohoto přepínače.



%----------------------------------------------------------------------------------------------------------

\section{Závěr}

V aplikaci bylo nalezeno několik, spíše drobnějších chyb především v parsování příkazů. Některé chyby již byly opraveny, některé teprve opravím a jistě existují chyby, zvláště v parserech, které nebyly ještě nalezeny. Ty budu muset opravit, až na ně uživatelé, budou-li kdy nějací, přijdou.

V simulátoru nebyly nalezeny žádné vážnější chyby v jeho datových strukturách a ve virtuální síti. Nebyly nalezeny ani žádné chyby, které by způsobovaly pád naší aplikace.

\chapter{Možná další vylepšení}

Simulátor by se dal v mnohých ohledech ještě vylepšit. Některé vylepšení by přinesla větší uživatelský komfort, jiná by simulátor učinila věrnější skutečné síti. Zde uvádím několik námětů k vylepšení:

% \begin{enumerate}
% \item Prvním by mohlo být grafické rozhraní pro tvorbu konfiguračního souboru (struktury počítačové sítě), kde by bylo možné přidávat počítače a propojení mezi nimi.                                                                                                                                                                     
% 
% \item Pro lepší ladění problémů při konfiguraci sítě by se jistě hodil i tcpdump. Tento program funguje jako analyzátor monitorující síťový provoz na daném rozhraní.
% 
% \item Aplikace podporuje pouze jeden síťový prvek - směrovač, a tak by bylo možné program rozšířit o~další prvky např. switch (přepínač) nebo bridge. Přidání těchto prvků by znamenalo plnohodnotnou implementaci 2. linkové vrstvy ISO/OSI modelu.
% 
% \item Dalším vylepšením by mohla být možnost propojit virtuální se skutečnou sítí. 
% 
% \item Aplikace de facto nezpracovává signály (Ctrl+C a Ctrl+Z), pouze je přeposílá operačnímu systému (tuto funkci zajišťuje rlwrap). Pokud bychom se chtěli vracet do privilegovaného módu pomocí Ctrl+Z, tak by bylo nutné implementovat vlastního klienta.
% \end{enumerate}

\begin{itemize}
\item Dle mého názoru je největším nedostatkem naší aplikace komunikace simulátoru s klientem. Především by bylo dobré najít nebo naprogramovat vlastního, třeba i grafického, klienta, pomocí kterého by se uuživatelé k simulátoru připojovali. Zvláště pod OS Windows ke připojování pomocí emulátoru cygwin velmi nepohodlné. Další obrovskou nevýhodou stávající komunikace je nemožnost posílat signály jako \verb|Ctrl+C| nebo \verb|Ctrl+Z|.
\item Uživateli by velmi pomohlo, kdyby si topologii sítě mohl vytvořit v nějakém, nejlépe grafickém programu. Také zobrazování topologie již vytvořené sítě by bylo velmi užitečným vylepšením.
\item Simulátor nepodporuje žádné síťové prvky kromě směrovačů na síťové vrstvě ISO/OSI modelu, pořádně není implementována ani linková vrstva tohoto modelu. Na síťovém rozhraní může být zatím jen jedna IP adresa. Pro potřeby předmětu PSI nejsou tyto funkcionality důležité, ale pro jiné účely by takovéto vylepšení bylo asi nutné. V tomto ohledu jsem ale realista a myslím si, že toto vylepšení nebude asi nikdy potřeba.
\item Některé linuxové příkazy nejsou zcela věrné příkazům na skutečném linuxu. Bylo by dobré opravit například parser u příkazů \verb|ifconfig|, \verb|ping| nebo \verb|traceroute|.
\end{itemize}

% V~této práci se podařilo navrhnout a implementovat simulátor virtuální počítačové sítě cisco. Ve spojení s~kolegovu částí je tento program schopen úspěšně odsimulovat virtuální síť složenou ze směrovačů cisco a z~počítačů založených na jádru linux. V~takové síti je možné otestovat znalosti studentů týkajících se konfigurace rozhraní, statického směrování, dynamického a statického překladu adres. Z~výsledků uživatelského testování vyplývá, že provedení takovýchto testů smysl má.
% 
% Nepřipravenost studentů na laboratorních cvičení z~předmětu Y36PSI Počítačové sítě je v~současné době problém. Obyčejní studenti zpravidla nemají přístup ke směrovačům od Cisco Systems, a tak by nasazení této aplikace jako studijní pomůcky mohlo vést ke zvýšení připravenosti těchto studentů\footnote{Za předpokladu, že studenti budou mít motivaci k~získání bodů ze cvičení.} na laboratorní cvičení.


\chapter{Závěr}

V této práci se mi společně s mým kolegou Stanislavem Řehákem povedlo navrhnout a implementovat jednoduchý síťový simulátor použitelný pro výukové účely předmětu PSI. Studenti, kteří nemají žádné zkušenosti s konfigurací síťových prvků, mají možnost vyzkoušet si konfiguraci na svém vlastním počítači ještě před bodovanou laboratorní úlohou. Informace o průchodu paketu, vypisované aplikací, můžou být studentům velmi užitečné při hledání chyb v konfiguraci.  



%*****************************************************************************
% Seznam literatury je v samostatnem souboru reference.bib. Ten
% upravte dle vlastnich potreb, potom zpracujte (a do textu
% zapracujte) pomoci prikazu bibtex a nasledne pdflatex (nebo
% latex). Druhy z nich alespon 2x, aby se poresily odkazy.

% originally following specification for bibliography formating was used
%\bibliographystyle{abbrv}

% Here is an improvment by Petr Dlouhy (April 2010).
% It is mainly for supervisors who expect Czech fomrating rules for references
% Additional feature is live url addresses to sources from your pdf file
% It requires the file csplainnat.bst (included in this sample zipfile).

\bibliographystyle{csplainnat}

%bibliographystyle{plain}
%\bibliographystyle{psc}
{
%JZ: 11.12.2008 Kdo chce mit v techto ukazkovych odkazech take odkaz na CSTeX:
\def\CS{$\cal C\kern-0.1667em\lower.5ex\hbox{$\cal S$}\kern-0.075em $}
\bibliography{reference}
}

% M. Dušek radi:
%\bibliographystyle{alpha}
% kdy citace ma tvar [AutorRok] (napriklad [Cook97]). Sice to asi neni  podle ceske normy (BTW BibTeX stejne neodpovida ceske norme), ale je to nejprehlednejsi.
% 3.5.2009 JZ polemizuje: BibTeX neobvinujte, napiste a poskytnete nam styl (.bst) splnujici citacni normu CSN/ISO.

%*****************************************************************************
%*****************************************************************************
\appendix

%*****************************************************************************
\chapter{Seznam použitých zkratek}

\begin{description}
\item[ARP] Address Resolution Protocol
\item[ICMP] Internet Control Message Protocol
\item[IOS] Internetwork Operating System
\item[IP] Internet Protocol
\item[Java SE] Java Standart Edition
\item[JRE] Java Runtime Enviroment
\item[MAC] Media Access Control
\item[NAT] Network Address Translation
\item[NVT] Network Virtual Terminal
\item[OS] Operační Systém
\item[Y36PSI, PSI] Předmět Počítačové sítě na FEL
\end{description}

%*****************************************************************************
% \chapter{Instalační a uživatelská příručka}
% \textbf{\large Tato příloha velmi žádoucí zejména u softwarových implementačních prací.}
\chapter{Instalační a~uživatelská příručka}
% \textbf{\large Tato příloha velmi žádoucí zejména u~softwarových implementačních prací.}

\section{Instalační příručka}

Autorem této uživatelské příručky k našemu projektu je můj kolega Stanislav Řehák.

\subsection{OS Windows}
Požadavky:
\begin{enumerate}
 \item Java Runtime Environment\\
      http://www.java.com/en/download/

 \item
       \begin{itemize}
        \item Cygwin\\
    http://www.cygwin.com/setup.exe
	\item telnet\\
    nainstalovat baliček inetutils pod cygwinem
	\item zkontrolovat, zda je nainstalován nejnovejší balíček libreadline
       \end{itemize}


\end{enumerate}


% 1) Java Runtime Environment
%     http://www.java.com/en/download/
% 2) a) Cygwin
%     http://www.cygwin.com/setup.exe
%    b) telnet
%     nainstalovat baliček inetutils pod cygwinem
%    c) zkontrolovat, zda je nainstalován nejnovejší balíček libreadline

\noindent
Krok 2 lze vynechat stažením předpřipraveného archivu s Cygwinem se správnými balíčky.\\
<<tady dat odkaz>>

\noindent
Nebo rozbalením souboru bin/psimulator\_win.zip z přiloženého CD.


\subsection{OS Linux}
Záleží na distribuci, ale obecně lze říci, že tyto programy budou v repozitářích.
Pro Debian-based distribuci lze nainstalovat jedním příkazem:
\begin{verbatim}
aptitude install sun-java6-jre rlwrap telnet
\end{verbatim} 

Požadavky:
\begin{enumerate}
\item Java Runtime Environment\\
      \verb|aptitude sun-java6-jre|\\
      http://www.java.com/en/download/
\item rlwrap - nejlépe ve verzi 0.32+\\
    \verb|aptitude install rlwrap|\\
    http://utopia.knoware.nl/\~hlub/rlwrap/
\item telnet\\
    \verb|aptitude install telnet|
\end{enumerate}





\section{Uživatelská příručka} 

\subsection{Spuštění serveru} 

Ve složce se skriptem musí být soubor s definicí DTD, která popisuje strukturu XML souboru.
Celý server se nastartuje příkazem:
\begin{verbatim}
./start_server <config> <port> 
\end{verbatim} 
Kde <config> je XML soubor s nastavením sítě (počítače, rozhraní, ..).
Parametr <port> říká, na jakém portu se začnou vytvářet jednotlivé počítače z XML souboru.
Parametr <port> je volitelný, defaultně je nastaven na 4000.
Volitelný parametr \verb|-n| umožní načtení pouze kostry sítě a počítačů s rozhraními.

Po spuštění serveru se vypíše seznam počítačů a k nim přiřazených portů.
Dále se bude na standartní výstup vypisou různé servisní informace.

\subsection{Připojení klientů} 
Pro připojení na cisco počítač:
\begin{verbatim}
./cisco.sh <port> 
\end{verbatim} 

Na ciscu je implementován příkaz help (help\_en), který vypisuje seznam podporovaných příkazů.

Pro připojení na linux počítač:
\begin{verbatim}
./linux.sh <port>
\end{verbatim} 



%*****************************************************************************
\chapter{Obsah přiloženého CD}
%\textbf{\large Tato příloha je povinná pro každou práci. Každá práce musí totiž obsahovat přiložené CD. Viz dále.}


Přiložené CD obsahuje tyto soubory a adresáře:

\begin{verbatim}
readme.txt              informace o souborech na CD
text/                   adresář s textem práce
text/install.txt        instalační příručka
text/readme.txt         uživatelská příručka
bin/                    adresář se spustitelnými a konfiguračními soubory
bin/psimulator_lin.zip  zabalený archiv se spustitelnými soubory pro linux
                        (PSImulator bez programů rlwrap a telnet)
bin/psimulator_win.zip  zabalený archiv se spustitelnými soubory pro windows
                        (PSImulator + Cygwin)
javadoc/                dokumentace javadoc
source/                 zdrojové kódy
source/netbeans.zip     zabalené zdrojové kódy jako projekt do NetBeans
source/config           konfigurační soubory použité pro testování
\end{verbatim} 


% Může vypadat například takto. Váš seznam samozřejmě bude odpovídat typu vaší práce. (viz \cite{infodp}):
% 
% \begin{figure}[h]
% \begin{center}
% \includegraphics[width=14cm]{figures/seznamcd}
% \caption{Seznam přiloženého CD --- příklad}
% \label{fig:seznamcd}
% \end{center}
% \end{figure}
% 
% Na GNU/Linuxu si strukturu přiloženého CD můžete snadno vyrobit příkazem:\\ 
% \verb|$ tree . >tree.txt|\\
% Ve vzniklém souboru pak stačí pouze doplnit komentáře.
% 
% Z \textbf{README.TXT} (případne index.html apod.)  musí být rovněž zřejmé, jak programy instalovat, spouštět a jaké požadavky mají tyto programy na hardware.
% 
% Adresář \textbf{text}  musí obsahovat soubor s vlastním textem práce v PDF nebo PS formátu, který bude později použit pro prezentaci diplomové práce na WWW.

\end{document}

\chapter{Instalační a~uživatelská příručka}
% \textbf{\large Tato příloha velmi žádoucí zejména u~softwarových implementačních prací.}

\section{Instalační příručka}

Autorem této uživatelské příručky k našemu projektu je můj kolega Stanislav Řehák.

\subsection{OS Windows}
Požadavky:
\begin{enumerate}
 \item Java Runtime Environment\\
      http://www.java.com/en/download/

 \item
       \begin{itemize}
        \item Cygwin\\
    http://www.cygwin.com/setup.exe
	\item telnet\\
    nainstalovat baliček inetutils pod cygwinem
	\item zkontrolovat, zda je nainstalován nejnovejší balíček libreadline
       \end{itemize}


\end{enumerate}


% 1) Java Runtime Environment
%     http://www.java.com/en/download/
% 2) a) Cygwin
%     http://www.cygwin.com/setup.exe
%    b) telnet
%     nainstalovat baliček inetutils pod cygwinem
%    c) zkontrolovat, zda je nainstalován nejnovejší balíček libreadline

\noindent
Krok 2 lze vynechat stažením předpřipraveného archivu s Cygwinem se správnými balíčky.\\
<<tady dat odkaz>>

\noindent
Nebo rozbalením souboru bin/psimulator\_win.zip z přiloženého CD.


\subsection{OS Linux}
Záleží na distribuci, ale obecně lze říci, že tyto programy budou v repozitářích.
Pro Debian-based distribuci lze nainstalovat jedním příkazem:
\begin{verbatim}
aptitude install sun-java6-jre rlwrap telnet
\end{verbatim} 

Požadavky:
\begin{enumerate}
\item Java Runtime Environment\\
      \verb|aptitude sun-java6-jre|\\
      http://www.java.com/en/download/
\item rlwrap - nejlépe ve verzi 0.32+\\
    \verb|aptitude install rlwrap|\\
    http://utopia.knoware.nl/\~hlub/rlwrap/
\item telnet\\
    \verb|aptitude install telnet|
\end{enumerate}





\section{Uživatelská příručka} 

\subsection{Spuštění serveru} 

Ve složce se skriptem musí být soubor s definicí DTD, která popisuje strukturu XML souboru.
Celý server se nastartuje příkazem:
\begin{verbatim}
./start_server <config> <port> 
\end{verbatim} 
Kde <config> je XML soubor s nastavením sítě (počítače, rozhraní, ..).
Parametr <port> říká, na jakém portu se začnou vytvářet jednotlivé počítače z XML souboru.
Parametr <port> je volitelný, defaultně je nastaven na 4000.
Volitelný parametr \verb|-n| umožní načtení pouze kostry sítě a počítačů s rozhraními.

Po spuštění serveru se vypíše seznam počítačů a k nim přiřazených portů.
Dále se bude na standartní výstup vypisou různé servisní informace.

\subsection{Připojení klientů} 
Pro připojení na cisco počítač:
\begin{verbatim}
./cisco.sh <port> 
\end{verbatim} 

Na ciscu je implementován příkaz help (help\_en), který vypisuje seznam podporovaných příkazů.

Pro připojení na linux počítač:
\begin{verbatim}
./linux.sh <port>
\end{verbatim} 


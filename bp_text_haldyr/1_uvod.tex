\chapter{Úvod} \label{uvod}

% Úvod charakterizující kontext zadání, případně motivace.
% ----------
% Navrhněte a~implementujte aplikaci, která umožní vytvoření virtuální počítačové sítě, pro potřeby předmětu Y36PSI. Na
% systém se bude možno připojit s~pomocí telnetu. Z~pohledu uživatele se bude systém tvářit jako reálná síť. Zaměřte se
% především na konfiguraci systému Cisco. Systém bude podporovat příkazy potřebné ke konfiguraci síťových rozhraní,
% směrování a~překladu adres. Pro ověření správnosti konfigurace implementujte příkaz ping a~traceroute.

Úkolem této práce je navrhnout a implementovat aplikaci, která umožní vytvoření virtuální počítačové sítě pro předmět Y36PSI Počítačové sítě. Z~pohledu uživatele se systém musí tvářit jako reálná síť. Tento úkol byl rozdělen na dvě části: cisco a linux. Můj úkol je právě emulace Cisco IOS\footnote{Internetwork Operating System je operační systém používaný na směrovačích a přepínačích firmy Cisco Systems}. Na dnešním virtuálním trhu existuje celá řada programů pro virtualizaci sítě. Většina z~nich je však špatně dostupných (zejména kvůli příliš restriktivní licenci) nebo se nehodí potřebám předmětu Počítačové sítě. 

Vize je taková, že student si v~teple domova spustí tuto aplikaci a vyzkouší si konfiguraci na virtuálním ciscu, protože ke skutečnému nemá přístup. Zjistí, jak to funguje, a pak už jen přijde na cvičení předmětu a vše správně nakonfiguruje.
Tato práce je týmovým projektem, protože přesahuje rozsah jedné bakalářské práce. Byly vymezeny hranice, aby se tento úkol mohl rozdělit na dvě práce. Nakonec byla celá aplikace rozdělena na tři části. První je část společná, kde je implementováno jádro klient - server. Druhou část tvoří Cisco IOS, kterou se zabývá tato práce. Poslední část je platforma Linux, kterou zpracoval Tomáš Pitřinec v~bakalářské práci Simulátor virtuální počítačové sítě Linux.


\section{Wrapper směrovací tabulky}
Routovací tabulka, kterou implementoval kolega, je \uv{šitá} pro linux. Abych ji mohl použít, musel jsem vytvořit \verb|CiscoWrapper|, který bude linuxovou tabulku ovládat. Hlavní důvodem pro vytvoření nějakého wrapperu je skutečnost, že cisco svoji routovací tabulku vypočítává z~nahozených rozhraní a ze statických pravidel vložených uživatelem. Má tedy zvlášť datovou strukturu pro statická pravidla a pro samotnou routovací tabulku.

Mohl jsem si implementovat kompletně vlastní routovací tabulku, ale nějaké podobě wrapperu bych se stejně nevyhnul. Byla by to tedy zbytečná práce, která by navíc znamenala určité zdvojení kódu.

%------------------------------------------------------------------------------

\subsection{Směrovací tabulka}
Linuxová routovací tabulka je složena ze záznamů, kde každý z~nich je tvořen těmito položkami: adresát, brána a rozhraní. Existují (pro tuto práci důležité) dva typy záznamů:
\begin{itemize}
 \item záznam na bránu - příznak UG, při přidávání UG záznamu platí, že nově přidávaná brána musí být dosažitelná příznakem U~(= nějakým záznamem na rozhraní)
 \item záznam na rozhraní - příznak U, převážně záznamy od nahozených rozhraní
\end{itemize}

%------------------------------------------------------------------------------

\subsubsection{Výběr záznamů}
Pravidla jsou ukládána do routovací tabulky podle masky - ta je hlavním kritériem. Když je tedy potřeba rozhodnout, který záznam vybrat pro příchozí paket, tak se začne procházet tabulka a vratí se první záznam, kde se shoduje adresát záznamu s~adresátem paketu. Tím zajistíme požadavek cisca, aby se směrovalo vždy podle adresáta s~nejvyšším počtem jedniček v~masce.

\subsection{Wrapper}
\verb|CiscoWrapper| v~podstatě kopíruje datové struktury linuxové routovací tabulky a přidává několik obslužných metod. Největším oříškem byla správná aktualizace routovací tabulky na základě wrapperu. Wrapper si pamatuje statické směrovací záznamy a dle nahozených rozhraní generuje správné záznamy do routovací tabulky. 

\subsubsection{Statické záznamy}
Statické směrovací záznamy se zadávají v~privilegovaném módu přes příkaz \\\verb|ip route <adresa> <maska> <brána OR rozhraní>|. Záznam se nepřidá pouze v~případě neexistujícího rozhraní a adresy ze zakázaného rozsahu. Do zakázaného rozsahu patří IP adresy ze třídy D a E. Třídy adres jsou popsány v~následujícím odstavci. V~dohledné době se začnou uvolňovat IP adresy ze třídy E kvůli nedostatku IPv4 adres. Až se tak stane, tak Cisco bude muset vydat aktualizaci svého IOSu, protože v~něm momentálně nelze přiřazovat adresy z~tohoto rozsahu.
\begin{verbatim}
Třída  Začátek 1. bajt  Maska           Sítí        Stanic v~každé síti
A~0       0–127    255.0.0.0       126         16 777 214
B      10      128-191  255.255.0.0     16384       65534
C      110     192-223  255.255.255.0   2 097 152   254
D      1110    224-239  multicast
E      1111    240-255  vyhrazeno jako rezerva
\end{verbatim}
Upravený výpis s~Wikipedie \cite{wiki:ip} a ověřený z~webu organizace IANA \cite{iana}.

\paragraph{}
Když se přidává záznam s~nedosažitelnou bránou, tak IOS nevypíše žádnou chybovou hlášku (narozdíl od linux, který vypíše \verb|SIOCADDRT: No such process|). Záznam se uloží do wrapperu a je přidán do routovací tabulky ve chvíli, kdy bude dosažitelný. 

Při jakékoliv změně IP adresy na rozhraní či změně statických záznamů se smaže celá routovací tabulka a spustí se aktualizační funkce \verb|update|. Ta nejdříve přidá záznamy na rozhraní (dle nastavených adres vsech rozhraních\footnote{V kapitole \ref{prijmiEthernetove} jsem takovému chování říkal \uv{samo-nastavující záznamy.}}) a pak začne postupně propočítávat jednotlivé záznamy z~wrapperu. Záznam na rozhraní je přidán automaticky pokud výstupní rozhraní není shozené. Záznam na bránu se hledá přes rekurzivní metodu \\\verb|najdiRozhraniProBranu|.

V~mé implementaci je zabudována ochrana proti smyčkám u~záznamů na bránu, která limituje délku takového řetězu na 100 záznamů na bránu.

Statická pravidla lze vypsat v~privilegovaném módu příkazem \verb|show running-config| a generovaná pravidla (tedy obsah routovací tabulky) přes příkaz \verb|show ip route|.

\subsection{Vlastnosti}
Při testování školního cisca jsem narazil na zajímavé vlastnosti Cisco IOS. Přidával jsem postupně různá statická pravidla a nechal si vypisovat stav routovací tabulky.
Nejdříve jsem vložil tyto záznamy v~konfiguračním módu:
\begin{verbatim}
ip route 0.0.0.0 0.0.0.0 FastEthernet0/0
ip route 3.3.3.0 255.255.255.0 2.2.2.2
ip route 8.0.0.0 255.0.0.0 9.9.9.254
ip route 13.0.0.0 255.0.0.0 6.6.6.6
ip route 18.18.18.0 255.255.255.0 51.51.51.9
ip route 51.51.51.0 255.255.255.0 21.21.21.244
ip route 172.18.1.0 255.255.255.252 FastEthernet0/0
ip route 192.168.9.0 255.255.255.0 2.2.2.2
\end{verbatim}

Výpis routovací tabulky přes příkaz \verb|show ip route|:
\begin{verbatim}
     51.0.0.0/24 is subnetted, 1 subnets
S~51.51.51.0 [1/0] via 21.21.21.244
     18.0.0.0/24 is subnetted, 1 subnets
S~18.18.18.0 [1/0] via 51.51.51.9
     3.0.0.0/24 is subnetted, 1 subnets
S~3.3.3.0 [1/0] via 2.2.2.2
     21.0.0.0/24 is subnetted, 1 subnets
C       21.21.21.0 is directly connected, FastEthernet0/0
S~192.168.9.0/24 [1/0] via 2.2.2.2
     172.18.0.0/30 is subnetted, 1 subnets
S~172.18.1.0 is directly connected, FastEthernet0/0
S~8.0.0.0/8 [1/0] via 9.9.9.254
S~13.0.0.0/8 [1/0] via 6.6.6.6
     192.168.2.0/30 is subnetted, 1 subnets
C       192.168.2.8 is directly connected, FastEthernet0/1
S*   0.0.0.0/0 is directly connected, FastEthernet0/0
\end{verbatim} 

Potom jsem smazal defaultní záznam \verb|0.0.0.0 0.0.0.0 FastEthernet0/0| a znovu jsem pořídíl výpis:
\begin{verbatim}
     51.0.0.0/24 is subnetted, 1 subnets
S~51.51.51.0 [1/0] via 21.21.21.244
     18.0.0.0/24 is subnetted, 1 subnets
S~18.18.18.0 [1/0] via 51.51.51.9
     3.0.0.0/24 is subnetted, 1 subnets
S~3.3.3.0 [1/0] via 2.2.2.2
     21.0.0.0/24 is subnetted, 1 subnets
C       21.21.21.0 is directly connected, FastEthernet0/0
S~192.168.9.0/24 [1/0] via 2.2.2.2
     172.18.0.0/30 is subnetted, 1 subnets
S~172.18.1.0 is directly connected, FastEthernet0/0
S~8.0.0.0/8 [1/0] via 9.9.9.254
S~13.0.0.0/8 [1/0] via 6.6.6.6
     192.168.2.0/30 is subnetted, 1 subnets
C       192.168.2.8 is directly connected, FastEthernet0/1
\end{verbatim} 
Jak je vidět, tak se defaultní záznam opravdu smazal (záznam \verb|S*| opravdu chybí). Po opětovném spuštění příkazu \verb|show ip route| po cca 20 vteřinách vypadá výpis následovně:
\begin{verbatim}
     51.0.0.0/24 is subnetted, 1 subnets
S~51.51.51.0 [1/0] via 21.21.21.244
     18.0.0.0/24 is subnetted, 1 subnets
S~18.18.18.0 [1/0] via 51.51.51.9
     21.0.0.0/24 is subnetted, 1 subnets
C       21.21.21.0 is directly connected, FastEthernet0/0
     172.18.0.0/30 is subnetted, 1 subnets
S~172.18.1.0 is directly connected, FastEthernet0/0
     192.168.2.0/30 is subnetted, 1 subnets
C       192.168.2.8 is directly connected, FastEthernet0/1
\end{verbatim} 

Obsah routovací tabulky se dramaticky změnil. Dlouho jsem si lámal hlavu čím to je způsobeno. Ptal jsem se spolužáků co mají CCNA\footnote{Cisco Certified Network Associate} certifikáty a nikdo mi neuměl vysvětil, jak je možné, že se obsah routovací tabulky může měnit v~čase bez nějaké třetí osoby. Dokonce jsem u~známého pouštěl \verb|Cisco Packet Tracer| a zkoušel jsem stejné příkazy, ale nebyl jsem schopen to přes tento simulátor zreprodukovat.

Nakonec jsem přišel na to, že školní cisco je natolik pomalé, že není schopno propočítat routovací tabulku v~reálném čase, a tak aktualizuje tabulku s~několika vteřinovými prodlevami. Prodleva se pohybovala v~závislosti na počtu záznamů v~rozmezí 5-40 vteřin, zkoušel jsem ale přidat maximálně asi 15 záznamů, protože pak už je výpis routovací tabulky začíná být nepříjemně nepřehledný. Při vyšším počtu záznamů by se prodleva zřejmě zvyšovala. 

\subsection{Odchylky}
Při implementaci wrapperu jsem se několikrát odchýlil od skutečného cisca:

\begin{enumerate}
 \item Prodlevu v~aktualizaci routovací tabulky jsem neimplementoval, protože studenti na ni nemohou téměř narazit. A~když si náhodou student všimne, že obsah tabulky nesedí, tak příkaz pro výpis zpravidla spustí znova a vše bude už v~pořádku. Navíc jiná cisca mohou být mnohem rychlejší než ty školní a když takovou vlastnost nemá ani oficiální simulátor, tak asi nemá smysl to implementovat zde.

 \item Stanovil jsem limit 100 statických pravidel na bránu propojených do jednoho dlouhého řetězu. Nepřepokládám, že by bylo něco takového potřeba, 100 záznamů by ale mělo být víc než dost. Příklad velmi krátkého řetězu o třech záznamech:
\begin{verbatim}
ip route 4.4.4.0 255.255.255.0 6.6.6.6
ip route 6.6.6.0 255.255.255.0 8.8.8.8
ip route 8.8.8.0 255.255.255.0 9.9.9.9
..
\end{verbatim} 

%  \item Jednotlivé záznamy při výpisu routovací tabulky bývají sdruženy do nadsítí. Mnohdy se takto ale vůbec nesdružuje, 
% \begin{verbatim}
%      147.132.2.0/24 is variably subnetted, 2 subnets, 2 masks
% C       147.132.2.64/26 is directly connected, FastEthernet0/0
% C       147.132.2.128/25 is directly connected, FastEthernet0/1
% S*   0.0.0.0/0 [1/0] via 147.132.2.254
% \end{verbatim} 

\end{enumerate}



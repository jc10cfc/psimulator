\chapter{Možná vylepšení}
% Diskuse dalšího možného pokračování práce.

V~tomto projektu je možné pokračovat mnohými vylepšeními, zde jsou některé z~nich:

\begin{enumerate}
\item Prvním by mohlo být grafické rozhraní pro tvorbu konfiguračního souboru (struktury počítačové sítě), kde by bylo možné přidávat počítače a propojení mezi nimi.                                                                                                                                                                     

\item Pro lepší ladění problémů při konfiguraci sítě by se jistě hodil i tcpdump. Tento program funguje jako analyzátor monitorující síťový provoz na daném rozhraní.

\item Aplikace podporuje pouze jeden síťový prvek - směrovač, a tak by bylo možné program rozšířit o~další prvky např. switch (přepínač) nebo bridge. Přidání těchto prvků by znamenalo plnohodnotnou implementaci 2. linkové vrstvy ISO/OSI modelu.

\item Dalším vylepšením by mohla být možnost propojit virtuální se skutečnou sítí. 

\item Aplikace de facto nezpracovává signály (Ctrl+C a Ctrl+Z), pouze je přeposílá operačnímu systému (tuto funkci zajišťuje rlwrap). Pokud bychom se chtěli vracet do privilegovaného módu pomocí Ctrl+Z, tak by bylo nutné implementovat vlastního klienta.
\end{enumerate}

% HOTOVO:
% mozna vylepseni prepsat do vet
% zkontrolovat experimenty
% abstrakt
% zaver - snad

%TODO: 
% projit celou praci znova

% uzivatelska prirucka
% zatezove testy - jestli je zrusim, tak zrusit i reference v specifikaci
% --
% poslat Janicce




% stare jen v bodech:
% \begin{itemize}
% \item grafické \uv{klikátko} pro tvorbu konfiguračního souboru (struktury počítačové sítě)
% \item tcpdump - analyzátor monitorující síťový provoz na daném rozhraní
% \item podpora přepínače (switch) - implementace 2. linkové vrstvy síťového ISO/OSI modelu
% \item propojení s reálnou sítí - zabalování a rozbalování paketů do IP protokolu
% \item zpracovaní signalů Ctrl+C, Ctrl+Z - implementace vlastního klienta
% \end{itemize} 